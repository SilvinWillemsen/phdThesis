\chapter{2D Systems}\label{ch:2Dsyst}


In this work, it is mainly used to model a simplified body in papers...


State variable $u = u(x,y,t)$  where $t\geq 0$ and $(x,y) \in \D$ where $\D$ is 2-dimensional. The state variable can then be discretised according to $u(x, y, t) \approx \ulmn$ with space $x = lh$ and $y = mh$ and time $t = nk$ and $k=1/\fs$. For simplicity the grid spacing in both the $x$ and $y$ directions are set to be the same but could be different.

In continuous time the  operators:
\begin{equation}
    \Delta = \pxx + \pyy
\end{equation}


The same shift operators as defined in Chapter \ref{ch:FDTD} can be applied to grid function $\ulmn$. Additional ones are
\begin{equation}
    e_{y+}\ulmn = u_{l, m+1}^n,\quad \text{and}\quad e_{y-}\ulmn= u_{l, m-1}^n.
\end{equation}


\section{2D Wave Equation}
The 2D wave equation be used to model an ideal membrane such as done in 

\begin{equation}\label{eq:2DwavePDE}
    \ptt u = c^2\Delta u
\end{equation}
where $c = \sqrt{T/\rho H}$ is the wavespeed (in m/s) $T$ is the tension per unit length\todo{check whether this is right..} applied to the boundary (in N/m), material density

\section{Thin plate}
Used in \citeP[A], \citeP[B], \citeP[D] and \citeP[E]
biharmonic operator, Laplacian in \eqref{eq:laplacian} applied to itself.
\begin{equation}\label{eq:platePDE}
    \rho H \ptt u = -D\Delta\Delta u
\end{equation}
where $D = EH^3/12(1-\nu^2)$
\section{Stiff membrane}
Combination between Eqs. \eqref{eq:2DwavePDE} and \eqref{eq:platePDE}
\begin{equation}\label{eq:2DwavePDE}
    \rho H \ptt u = T\Delta u
\end{equation}
\citeP[F]