%  A simple AAU PhD thesis template (collection of papers).
%  2016-08-01 v. 1.3.1
%  Copyright 2012-2016 by Jesper Kjær Nielsen <jkn@es.aau.dk>
%
%  This is free software: you can redistribute it and/or modrighte{caption}
\captionsetup{%
  font=footnotesize,% set font size to footnotesize
  labelfont=bf % bold label (e.g., Figure 3.2) font
}
% Make the standard latex tables look so much better
\usepackage{array,booktabs}
\usepackage{tabularx}
\usepackage{multirow}
% Enable the use of frames around, e.g., theorems
% The framed package is used in the example environment
\usepackage{framed}
% Create beautiful plots using TikZ and PGFPLOTS
\usepackage{tikz,pgfplots}
%%%%%%%%%%%%%%%%%%%%%%%%%%%%%%%%%%%%%%%%%%%%%%%%
% Mathematics
% http://en.wikibooks.org/wiki/LaTeX/Mathematics
%%%%%%%%%%%%%%%%%%%%%%%%%%%%%%%%%%%%%%%%%%%%%%%%
% Defines new environments such as equation,
% align and split 
\usepackage{amsmath}
% Adds new math symbols
\usepackage{amssymb}
\usepackage{mathtools}

% Use theorems in your document
% The ntheorem package is also used for the example environment
% When using thmmarks, amsmath must be an option as well. Otherwise \eqref doesn't work anymore.
\usepackage[framed,amsmath,amsthm,thmmarks]{ntheorem}

%%%%%%%%%%%%%%%%%%%%%%%%%%%%%%%%%%%%%%%%%%%%%%%%
% Page Layout and appearance
% http://en.wikibooks.org/wiki/LaTeX/Page_Layout
%%%%%%%%%%%%%%%%%%%%%%%%%%%%%%%%%%%%%%%%%%%%%%%%
% Change margins, papersize, etc of the document
\usepackage[
  paperwidth=17cm, % width of a page
  paperheight=24cm, % height of a page
  outer=2.5cm, % right margin on an odd page
  inner=2.5cm, % left margin on an odd page
  top=2.5cm, % top margin
  bottom=2.5cm % bottom margin
  ]{geometry}
% Enable the crop package if you want to print on a4 paer
%\usepackage[a4,cam,center]{crop}
% Modify how \chapter, \section, etc. look
% \renewcommand{\thesection}{\arabic{section}}
% The titlesec package is very configureable
\usepackage{titlesec}
\titleformat*{\section}{\normalfont\Large\bfseries\color{aaublue}}
\titleformat*{\subsection}{\normalfont\large\bfseries\color{aaublue}}
\titleformat*{\subsubsection}{\normalfont\normalsize\bfseries\color{aaublue}}
%\titleformat*{\paragraph}{\normalfont\normalsize\bfseries\color{aaublue}}
%\titleformat*{\subparagraph}{\normalfont\normalsize\bfseries\color{aaublue}}
% Change some default names
\addto\captionsenglish{%this line is required when using the babel package
  % \renewcommand\appendixname{Paper} % change Appendix to Paper
  \renewcommand\bibname{References} % change Bibliography to references
  \renewcommand\figurename{Fig.} % change Figure to Fig.
}

% Change the headers and footers
\usepackage{fancyhdr}
\pagestyle{fancy}
\fancyhf{} %delete everything
\renewcommand{\headrulewidth}{0pt} %remove the horizontal line in the header
\fancyhead[CE]{\color{aaublue}\small\nouppercase\leftmark} %even page - chapter title
\fancyhead[CO]{\color{aaublue}\small\nouppercase\rightmark} %uneven page - section title
\fancyfoot[CE,CO]{\thepage} %page number on all pages
% Do not stretch the content of a page. Instead,
% insert white space at the bottom of the page
\raggedbottom
% Enable arithmetics with length. Useful when
% typesetting the layout.
\usepackage{calc}
% fix the marginpar command so it is always on the correct side
\usepackage{mparhack}


\usepackage{longtable}

%%%%%%%%%%%%%%%%%%%%%%%%%%%%%%%%%%%%%%%%%%%%%%%%
% Bibliography
% http://en.wikibooks.org/wiki/LaTeX/Bibliography_Management
%%%%%%%%%%%%%%%%%%%%%%%%%%%%%%%%%%%%%%%%%%%%%%%%
% Bibliography for each chapter
% \usepackage[sectionbib]{chapterbib}
% Custom bibliograhy - used in the list of papers
\usepackage[resetlabels]{multibib}
% \usepackage[style=ieee, backend=biber]{biblatex}
% \addbibresource{bib/mybib.bib}

\newcites{main}{References}
\newcites{A}{Main Publications}
\newcites{B}{Publications with a Supervisory Role}
\newcites{C}{Miscellaneous Publications}
% \renewcommand*{\bibliographyitemlabel}{[\alph{enumiv}]}

% \makeatletter
% \newrobustcmd*{\mknumAlph}[1]{%
%   \begingroup
%   \blx@tempcnta=#1\relax
%   \ifnum\blx@tempcnta>702 %
%   \else
%     \ifnum\blx@tempcnta>26 %
%       \advance\blx@tempcnta\m@ne
%       \divide\blx@tempcnta26\relax
%       \blx@numalph\blx@tempcnta
%       \multiply\blx@tempcnta26\relax
%       \blx@tempcnta=\numexpr#1-\blx@tempcnta\relax
%     \fi
%   \fi
%   \blx@numAlph\blx@tempcnta
%   \endgroup}
% \def\blx@numAlph#1{%
%   \ifcase#1\relax\blx@warning@entry{Value out of range}\number#1\or
%   A\or B\or C\or D\or E\or F\or G\or H\or I\or J\or K\or L\or M\or
%   N\or O\or P\or Q\or R\or S\or T\or U\or V\or W\or X\or Y\or Z\else
%   \blx@warning@entry{Value out of range}\number#1\fi}
% \makeatother

% \DeclareFieldFormat{labelnumber}{\ifkeyword{mine}{\mknumAlph{#1}}{#1}}



% Change [1,2,3,4] into [1-4]
% \usepackage{cite}

%%%%%%%%%%%%%%%%%%%%%%%%%%%%%%%%%%%%%%%%%%%%%%%%
% Misc
%%%%%%%%%%%%%%%%%%%%%%%%%%%%%%%%%%%%%%%%%%%%%%%%
% Add bibliography and index to the table of
% contents
\usepackage{tocbibind}
% Enable subappendices
\usepackage{appendix}
\renewcommand{\setthesection}{\Alph{section}} % remove the chapter numbering
% Add the command \pageref{LastPage} which refers to the
% page number of the last page
\usepackage{lastpage}
% Add notes to in your document
\usepackage[
%  disable, %turn off todonotes
  colorinlistoftodos, %enable a coloured square in the list of todos
  textwidth=2cm, %set the width of the todonotes
  textsize=scriptsize, %size of the text in the todonotes
  ]{todonotes}
\setlength{\marginparwidth}{2cm}

%%%%%%%%%%%%%%%%%%%%%%%%%%%%%%%%%%%%%%%%%%%%%%%%
% Hyperlinks
% http://en.wikibooks.org/wiki/LaTeX/Hyperlinks
%%%%%%%%%%%%%%%%%%%%%%%%%%%%%%%%%%%%%%%%%%%%%%%%
% Enable hyperlinks and insert info into the pdf
% file. Hypperref should be loaded as one of the 
% last packages
\usepackage{hyperref}
\hypersetup{%
	pdfpagelabels=true,%
	plainpages=false,%
	pdfauthor={Author},%
	pdftitle={Title},%
	pdfsubject={Subject},%
	bookmarksnumbered=true,%
	colorlinks=true,%
	citecolor=aaublue,%
	filecolor=aaublue,%
	linkcolor=aaublue,% you should probably change this to black before printing
	urlcolor=aaublue,%
	pdfstartview=FitH%
}

\usepackage{xcolor}
\def\SBcomment[#1]{\textcolor{Red}{#1}}
\def\SWcomment[#1]{\textcolor{Cyan}{#1}}
\def\MDcomment[#1]{\textcolor{Green}{#1}}
\def\SScomment[#1]{\textcolor{Bittersweet}{#1}}

\usepackage{cases}
\usepackage[final]{pdfpages}

\usepackage{dirtytalk}
\usepackage{subfig}

\usepackage{tikz}
\tikzset{>=latex}
\tikzstyle{block} = [draw,minimum size=0.5cm]
\usetikzlibrary{math,arrows,positioning,shapes.geometric, decorations.markings}

\usepackage{listings}
\usepackage{courier}

\usepackage[most]{tcolorbox}
\usepackage{tabstackengine}
\stackMath


\def\eqrefMatlab[#1]{%
    \hypersetup{linkcolor=[HTML]008400}%
    \color[HTML]{008400}{\texttt{(\ref{#1}})}
}

\renewcommand{\lstlistingname}{Algorithm}% 
\def\setlstCpp
{
  \lstset{ %
  backgroundcolor=\color{black!0},   % choose the background color; you must add \usepackage{color} or \usepackage{xcolor}
  basicstyle=\footnotesize\ttfamily,        % the size of the fonts that are used for the code
  breakatwhitespace=true,         % sets if automatic breaks should only happen at whitespace
  breaklines=true,                 % sets automatic line breaking
  captionpos=b,                    % sets the caption-position to bottom
  commentstyle=\color[HTML]{008400},    % comment style
  % escapeinside={\%*}{*)},          % if you want to add LaTeX within your code
  extendedchars=true,              % lets you use non-ASCII characters; for 8-bits encodings only, does not work with UTF-8
  frame=tb,	                   	   % adds a frame around the code
  keepspaces=true,                 % keeps spaces in text, useful for keeping indentation of code (possibly needs columns=flexible)
  keywordstyle=\color[HTML]{B82BA1},       % keyword style
  language=C++,                 % the language of the code (can be overrided per snippet)
  % otherkeywords={*,...},           % if you want to add more keywords to the set
  numbers=none,                    % where to put the line-numbers; possible values are (none, left, right)
  numbersep=5pt,                   % how far the line-numbers are from the code
  numberstyle=\tiny\color{black},%\noncopynumber, % the style that is used for the line-numbers
  rulecolor=\color{black},         % if not set, the frame-color may be changed on line-breaks within not-black text (e.g. comments (green here))
  showspaces=false,                % show spaces everywhere adding particular underscores; it overrides 'showstringspaces'
  showstringspaces=false,          % underline spaces within strings only
  showtabs=false,                  % show tabs within strings adding particular underscores
  stepnumber=1,                    % the step between two line-numbers. If it's 1, each line will be numbered
  stringstyle=\color[HTML]{D12F1B}, % string literal style
  tabsize=2,	                   % sets default tabsize to 2 spaces
  title=\lstname,                  % show the filename of files included with \lstinputlisting; also try caption instead of title
  columns=fixed,                    % Using fixed column width (for e.g. nice alignment),
  deletekeywords={*, float},            % if you want to delete keywords from the given language
  }
}

\def\setlstMAT
{
  \lstset{ %
    backgroundcolor=\color[HTML]{FCFDDB},   % choose the background color; you must add \usepackage{color} or \usepackage{xcolor}
    basicstyle=\footnotesize\ttfamily,        % the size of the fonts that are used for the code
    breakatwhitespace=false,         % sets if automatic breaks should only happen at whitespace
    breaklines=true,                 % sets automatic line breaking
    captionpos=b,                    % sets the caption-position to bottom
    commentstyle=\color[HTML]{008400},    % comment style
    escapeinside={\%*}{*)},          % if you want to add LaTeX within your code
    extendedchars=true,              % lets you use non-ASCII characters; for 8-bits encodings only, does not work with UTF-8
    frame=tb,	                   	   % adds a frame around the code
    keepspaces=true,                 % keeps spaces in text, useful for keeping indentation of code (possibly needs columns=flexible)
    keywordstyle=\color[HTML]{0000FF},       % keyword style
    language=MATLAB,                 % the language of the code (can be overrided per snippet)
    otherkeywords={...},           % if you want to add more keywords to the set
    numbers=left,                    % where to put the line-numbers; possible values are (none, left, right)
    numbersep=5pt,                   % how far the line-numbers are from the code
    numberstyle=\tiny\color{black},%\noncopynumber, % the style that is used for the line-numbers
    rulecolor=\color{black},         % if not set, the frame-color may be changed on line-breaks within not-black text (e.g. comments (green here))
    showspaces=false,                % show spaces everywhere adding particular underscores; it overrides 'showstringspaces'
    showstringspaces=false,          % underline spaces within strings only
    showtabs=false,                  % show tabs within strings adding particular underscores
    stepnumber=1,                    % the step between two line-numbers. If it's 1, each line will be numbered
    stringstyle=\color[HTML]{A100F4}, % string literal style
    tabsize=2,	                   % sets default tabsize to 2 spaces
    title=\lstname,                  % show the filename of files included with \lstinputlisting; also try caption instead of title
    columns=fixed,                   % Using fixed column width (for e.g. nice alignment)
    deletekeywords={pi, zeros, plot, round, ceil, floor, cos, sin},            % if you want to delete keywords from the given language
    morestring=[b]"
  }
}

