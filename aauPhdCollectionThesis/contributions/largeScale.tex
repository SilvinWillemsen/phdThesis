\chapter{Large Scale Modular Physical Models}\label{ch:largeScale}
This chapter provides an extended summary for the work presented in the papers ``Real-Time Control of Large-Scale Modular Physical Models using the Sensel Morph'' \citeP[A] and ``Physical Models and Real-Time Control with the Sensel Morph'' \citeP[B].  Paper \citeP[A] presents the work done on various physical models connected by nonlinear springs using three instruments as case studies: the esraj (bowed sitar), the hammered dulcimer and the hurdy gurdy. The implementations and a video showcasing the hurdy gurdy can be found online.\footnote{\url{https://github.com/SMC-AAU-CPH/ConnectedElements/releases/tag/v5.0}}\textsuperscript{,}\footnote{\url{https://youtu.be/BkxLji2ap1w}} Paper \citeP[A] follows \cite{theBible} and \cite{Bilbao2009Modular} and uses `scaling' (see Section \ref{sec:1DwaveContTime}). To relate the paper to the theory presented in this thesis, this chapter presents the models in a non-scaled, dimensional form. The eventual implementation of the models is equivalent. Then, a summary of the remaining parts of the paper is provided, including descriptions of the instruments.
% Furthermore, this chapter will build on the contents of paper \citeP[A] by providing more details on the implementation.\todo{will it though?}

\section{Physical models}\label{sec:modelsLargeScale}
All instruments use multiple instances of the stiff string presented in Chapter \ref{ch:stiffString} and one instance of the thin plate presented in Section \ref{sec:thinPlate}. The latter was used as a simplified instrument body for the resulting simulations (see Section \ref{sec:largeScaleInstruments}). Theory on connections can be found in Chapter \ref{ch:connections}, and information on the string-plate connection, specifically, is presented in Section \ref{sec:stringPlateConnection}.

Consider a set of strings, where the transverse displacement of string $s$ is described by $u_s = u_s(\chi_s, t)$ (in m) defined for $t\geq 0$ and $\chi_s \in \D_s$ for domain $\D_s = [0, L_s]$ and length $L_s$ (in m). Notice that every string is defined for a separate coordinate system $\chi_s$. In the following, spatial derivatives $\partial_{\chi_s}$ are the same as those described in Section \ref{sec:FDoperators}, but with respect to coordinate $\chi_s$. The PDE of string $s$ with an external connection force is defined as %(after division by $\rho_sA_s$)
\begin{equation}
    \begin{aligned}
    \ptt u_s = c_s^2 \partial_{\chi_s\chi_s} u_s &- \kappa_s^2 \partial_{\chi_s\chi_s\chi_s\chi_s} u_s - 2 \szX[s]\pt u_s\\
    &\quad + 2 \soX[s] \partial_{\chi_s\chi_s} u_s - \delta(\chi_s - \chi_{s, \ctxt})\frac{f_s}{\rho_sA_s},
    \end{aligned}
\end{equation}
where spatial Dirac delta function $\delta(\chi_s - \chi_{s,\ctxt})$ (in m$^{-1}$) localises the connection force between the string $s$ and the plate to connection location $\chi_{s, \ctxt}$. Other parameters are as defined in Eq. \eqref{eq:stiffStringPDE} but have a subscript $s$ to denote that they can be different for each strings. 

As all connections in the implementation are between an individual string and the plate, the PDE of the thin plate in Eq. \eqref{eq:platePDE} can be extended to 
\begin{equation}
    \begin{aligned}
    \!\!\!\!\ptt w = -\kappa_\ptxt^2\Delta\Delta w &\!-\! 2\szX[\ptxt]\pt w\! +\! 2\soX[\ptxt] \pt\pxx w+ \!\sum_s\delta(x \!-\! x_{\ctxt,s}, y \!-\! y_{\ctxt,s})\frac{f_s}{\rho_\ptxt H},\!\!\!\!\!\!
    \end{aligned}
\end{equation}
where 2D spatial Dirac delta function $\delta(x_s - x_{s,\ctxt}, y_s - y_{s,\ctxt})$ (in m$^{-2}$) localises the connection force of between the plate and string $s$ to coordinate $(x_s, y_s)$ on the plate. Other parameters are as defined in Eq. \eqref{eq:platePDE}. 

Finally, the connection force between the plate and string $s$ is defined as a nonlinear damped spring (see Eq. \ref{eq:nonlinearForce})
\begin{equation}
    f_s = K_1\eta_s + K_3\eta_s^3 + R \dot \eta_s,
\end{equation}
where
\begin{equation}
    \eta_s = u_s(\chi_{\ctxt,s}, t) - w(x_{\ctxt,s}, y_{\ctxt,s}, t)
\end{equation}
is the relative displacement between string $s$ and the plate at their respective connection locations. Notice that the plate is placed below the strings such that the sign of the force term is negative for the strings and positive for the plate.

\section{Implementation}
This section provides considerations for implementing the above models. Details on discretisation of the models and how to solve for $f_s$ are presented in Section \ref{sec:stringPlateConnection} and are not given here. 

The spatial Dirac delta functions are discretised using 0\thOrder spreading operators for simplicity (see Section \ref{sec:interpolationSpreading} (1D) and Section \ref{sec:interpolationSpreading2D} (2D)). Furthermore, the connection locations on the plate are implemented to be non-overlapping. Overlaps would require to solve a system of linear equations to obtain the connection forces (see e.g. \cite{Bilbao2009Modular}). Looking towards real-time implementation, an explicit solution for each connection is desired.

\section{Summary}
This section provides a summary of the instrument simulations presented in paper \citeP[A]. All instruments were implemented in real-time in C++ using the JUCE framework (see Chapter \ref{ch:realtime}). Finally, a summary of the results and the conclusion will be given.

\subsection{Instruments}\label{sec:largeScaleInstruments}
Using the setup presented in Section \ref{sec:modelsLargeScale}, various configurations inspired by real instruments have been made. The choices of simulated instruments were aimed at those containing many (sympathetic) strings.\footnote{Sympathetic strings -- apart from being friendly -- are strings that add resonances to the instrument without being excited directly.} Another condition was that no FDTD-based physical models existed in the literature at the time of writing the papers.

Three implementations inspired by real-life instruments were created and their setups are presented here. The implementations were controlled by a pair of Sensel Morph controllers (see Section \ref{sec:sensel}). The mapping between the controllers and the instruments is explained in papers \citeP[A] and \citeP[B].

\subsubsection{Esraj: bowed sitar}
The first instrument simulation was inspired by the \textit{esraj}: the bowed sitar. This instrument uses many strings, some of which are bowed and others are sympathetic strings that resonate when the instrument is played. As one can also interact with the latter, several strings in the implementation could be plucked as well. 

In total, 20 strings were implemented, all connected to a thin plate: 2 strings could be bowed, 5 strings could be plucked, and 13 strings are sympathetic. The bow was implemented using the static friction model presented in Section \ref{sec:staticFricMod} and the pluck was modelled as a time-varying raised cosine found in Section \ref{sec:timeVaryingRaisedCos}.

\subsubsection{Hammered dulcimer}
The hammered dulcimer, or santur, can be seen as an `open piano' where the player hammers several strings at once. In the implementation, 20 pairs of strings are implemented, and one in each pair is connected to the plate. This causes a slight detuning between the strings, resulting in a characteristic `chorus' effect exhibited by the instrument. To excite the strings, the time-varying strike presented in Section \ref{sec:timeVaryingRaisedCos} is used.

\subsection{Hurdy gurdy}
The hurdy gurdy is a bowed string instrument, that also uses sympathetic strings. Rather than a bow, the instrument uses a rosined wheel attached to a crank that bows the strings as it is turned. As for the esraj, the static friction model presented in Section \ref{eq:staticFriction} was used to implement the wheel. 

The instrument simulation consists of 5 bowed strings and 13 sympathetic strings, all connected to a plate. 

\subsection{Results and conclusion}
All instrument simulations were able to run in real time on a MacBook Pro with a 2.2 GHz Intel i7 processor.
Interaction with the implementations shows that when exciting one string, the connections with the plate cause other (sympathetic) strings to vibrate as well. Specifically, strings tuned to one of the harmonic partials of the excited string were found to resonate to a high degree. This phenomenon is consistent to real-world processes.

Finally, informal evaluations of the instruments were carried out on experts in the sound and music computing field, and showed that the mapping between the Sensels and the instruments, specifically the bowing interaction, was considered natural and intuitive.