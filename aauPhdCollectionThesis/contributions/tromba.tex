\chapter{Tromba Marina}\label{ch:tromba}
\section{Introduction}
The tromba marina is a bowed monochord instrument from medieval Europe. It has a long quasi-trapezoidal body and is unique due to its oddly-shaped bridge that the string rests on. The bridge is often called a `shoe' due to its shape and is free to rattle against the body in sympathy with the movement of the vibrating string. This rattling causes a sound with brass or trumpet-like qualities, hence the name \textit{tromba} which stems from the Italian word trumpet. The rarity of the instrument as well as its interesting physics makes it an ideal case for a physical modelling implementation.

\section{Physical Model}
Using linear (partial) differential operator $\L$ 
\begin{equation}
    \L q = 0
\end{equation}
where $q(\boldsymbol{x}, t)$
\subsection{Continuous}
\begin{equation}
    \L_\stxt = \rho_\stxt A \ptt - T\partial_\chi^2 + E_\text{s}I\partial_\chi^4+2\rho_\stxt A\szX[\stxt]\pt-2\rho_\stxt A\soX[\stxt]\pt\partial_\chi^2\ .
\end{equation}

\subsubsection{Complete system}
Test
\subsection{Discrete}

\section{Real-Time Implementation}

\subsection{Control using Sensel Morph}

\subsection{VR Application}
