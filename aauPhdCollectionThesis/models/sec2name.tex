\section{Section 2 name}\label{sec:ch2label}
Here is section 2. If you want to leearn \todo{I think this word is mispelled} more about \LaTeXe{}, have a look at \cite{Madsen2010}, \cite{Oetiker2010} and \cite{Mittelbach2005}.
\missingfigure{We need a figure right here!}

\subsection{Examples}
You can also have examples in your document such as in example~\ref{ex:simple_example}.
\begin{example}{An Example of an Example}
  \label{ex:simple_example}
  Here is an example with some math
  \begin{equation}
    0 = \exp(i\pi)+1\ .
  \end{equation}
  You can adjust the colour and the line width in the {\tt macros.tex} file.
\end{example}

\subsection{How does Subsections and Subsubsections Look?}
Well, like this
\subsubsection{This is a Subsubsection}
and this.

\paragraph{A Paragraph}
You can also use paragraph titles which look like this.

\subparagraph{A Subparagraph} Moreover, you can also use subparagraph titles which look like this\todo{Is it possible to add a subsubparagraph?}. They have a small indentation as opposed to the paragraph titles.

\todo[inline,color=green]{I think that a summary of this exciting chapter should be added.}

