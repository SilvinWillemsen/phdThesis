\section{Background and Motivation}
Simulating musical instruments using physical modelling -- as mentioned in Part \ref{part:introduction}\todo{check if is still true} -- allows for manipulations of the instrument that are impossible in the physical world. Examples of this are changes in material density or stiffness, cross-sectional area (1D), thickness (2D) and size. Apart from being potentially sonically interesting, there are examples in the physical world where certain aspects of the instrument are manipulated in real-time.

Tension in a string is changed when tuning it

Some artists even use this in their performances \cite{Gomm2011}.

A more relevant example is that of the trombone, where the size of the instrument is changed in order to play different pitches. Modelling this using FDTD methods would require

% \section{Issues with FDTD methods}


In his thesis, Harrison points out that in order to model the trombone, grid points need to be introduced 