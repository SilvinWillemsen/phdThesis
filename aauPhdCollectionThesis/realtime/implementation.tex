\chapter{Real-Time Implementation}
JUCE
Give overall structure of code


Implementation of the physical models
using FDTD methods

As mentioned in Chapter \ref{ch:physMod}, FDTD methods are used for high-quality and accurate simulations, rather than for real-time applications. This is due to their lack of simplifications.

Usually, \texttt{MATLAB} is used for simulating 

Here, an interactive application is considered real-time when
\begin{center}\it
    Control of the application generates or manipulates audio with no noticeable latency.
\end{center}

Also the application needs to be controlled continuously

Helps to (informally) evaluate the models by interacting with it in a natural way (rather than static parameters)

\section{MATLAB vs. C++}
It is usually a good idea to prototype a physical modelling application in MATLAB for several reasons:
\begin{itemize}
    \item Easier to debug
    \begin{itemize}
        \item Plotting functionality
        \item No need for memory handling
    \end{itemize}
    \item Instability due to programming errors 
\end{itemize}

\subsection{Speed}
Here is where the power of C++ 



\subsection{Syntax}
Indexing in
Matlab is 1-based, meaning that the index of a vector starts at 1. If \texttt{u} is a vector with 10 elements, the first element is retrieved as \texttt{u(1)} and the last as \texttt{u(10)}. C++, on the other hand, is 0-based and retrieving the first and last element of a size-10 vector happens through \texttt{u[0]} and \texttt{u[9]}respectively. 

\section{Do's and don'ts in Real-Time FD schemes}
Some of the things I learned (the hard way)...
\begin{itemize}
    \item Create a limiter
    \item Structure your application into classes 
    \item Use pointer switches
    \item Comment your code (hehe)
\end{itemize}

\subsubsection{Create a limiter}
Programming errors happen. To save your speakers, headphones or -- most importantly -- your ears, create a limiter. 

\setlstCpp
\begin{lstlisting}
double limit (double val)
{
    if (val < -1)
    {
        val = -1;
        return val;
    }
    else if (val > 1)
    {
        val = 1;
        return val;
    }
    return val;
}
\end{lstlisting}

\setlstMAT
\begin{lstlisting}
for  i = 0:lengthSound
    uNext = ...
end
\end{lstlisting}

\subsubsection{Use pointer switches}
One of the most important things in working with FD schemes for real-time audio applications is to 

\setlstMAT
\begin{lstlisting}
for n = 1:lengthSound
    ...
    uPrev = u;
    u = uNext;
end
\end{lstlisting}

In C++ this is done using
\setlstCpp
\begin{lstlisting}
double updateStates()
{
    double* uTmp = u[2];
    u[2] = u[1];
    u[1] = u[0];
    u[0] = uTmp;
}
\end{lstlisting}