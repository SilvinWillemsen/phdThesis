% mainfile: ../master.tex
\chapter*{Preface\markboth{Preface}{Preface}}\label{ch:preface}
\pagestyle{fancy}
\addcontentsline{toc}{chapter}{Preface}

Starting this PhD, I did not have a background in mathematics, physics or computer science, which were three equally crucial components in creating the result of this project. This is why I decided to write this thesis a bit more pedagogical than what could be expected. As I felt that the literature lacks a lot of intuition I wanted to give that to the reader. Some basic calculus knowledge is assumed.  

I wanted to show my learning process and (hopefully) explain topics such as \textit{Energy Analysis}, \textit{Stability Analysis}, etc. in a way that others lacking the same knowledge %(/ with the same background) 
will be able to understand.

Make physical modelling more accessible to the non-physicist.

\textit{Interested in physically impossible manipulations of now-virtual instruments.}

\vfill
\hfill Silvin Willemsen

\hfill Aalborg University, \today
