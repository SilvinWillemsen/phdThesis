% mainfile: ../master.tex
\chapter*{Preface\markboth{Preface}{Preface}}\label{ch:preface}
\pagestyle{fancy}
\addcontentsline{toc}{chapter}{Preface}

Starting this Ph.D. project, I did not have a background in mathematics, physics or computer science, which were three equally crucial components in creating the result of this project. After the initial steep learning curve of notation and terminology, I was surprised to find that the methods used for physical modelling are actually quite straightforward! 

Of course it should take a bit of time to learn these things, but 



Many concepts that seemed impossible at the beginning 

I feel that the literature lacks a lot of the intuition needed for readers without a background in any of these topics. Rather, much of the literature I came across assumes that the reader has a degree in at least one of the aforementioned topics. Stefan Bilbao's seminal work \textit{Numerical Sound Synthesis}, which is the most complete work to date describing how to physically model musical instruments using finite-difference time-domain methods says that "A strong background in digital signal processing, physics, and computer programming is essential."
Even though some basic calculus knowledge is assumed to understand the concepts used in this work, a degree in any of the aforementioned topics is (hopefully) unnecessary. \SWcomment[Furthermore, some experience with MATLAB and C++ is assumed for the code examples]

Some working titles: \textit{Physical Modelling for Dummies} \textit{Physical Modelling for the faint-hearted}

Also, I came across a lot of ``it can be shown that's without derivations. This is why I decided to write this work a bit more pedagogical, and perhaps more elaborate than what could be expected. 

I believe that anyone with some basic skills in mathematics and programming is able to create a simulation based on physics within a short amount of time, given the right tools, which I hope that this dissertation could be.  

The knowledge dissemination of this dissertation is thus not only limited to the research done and publications made over the course of the project, but also its pedagogical nature hopefully allowing future (or past) students to benefit from.

As with a musical instrument itself, a low entry level, a gentle learning curve along with a high virtuosity level is desired. Take a piano, for instance. Most will be able to learn a simple melody — such as “Fr\`ere Jacques” — in minutes, but to become virtuous requires years of practice.

This is the way I wanted to write this dissertation: easy to understand the basic concepts, but many different aspects touched upon to allow for virtuosity. Hopefully by the end, the reader will at least grasp some highly complex concepts in the fields of mathematics, physics and computer science (which will hopefully take less time than it takes to become virtuous at playing the piano). 

As Smith states in his work \textit{Physical Audio Signal Processing} \cite{Smith2010} ``All we need is Newton'', and indeed, all Newton's laws of motion will make their appearance in this document.


I wanted to show my learning process and (hopefully) explain topics such as \textit{Energy Analysis}, \textit{Stability Analysis}, etc. in a way that others lacking the same knowledge %(/ with the same background) 
will be able to understand.

Make physical modelling more accessible to the non-physicist. Also supports reproducibility of science and lowers the entry level   


\textit{Interested in physically impossible manipulations of now-virtual instruments.}


\SWcomment[Could be fun to include this :) :] ``I'm a musician. Will I be out of a job if you keep making physical models?'' Physical modelling is not here to replace the original instruments and the musicians playing them. Instead, it can be used as a tool to understand the physics of existing instruments and possibly go beyond. Simulated instruments are not restricted by physics anymore and could provide new ways of expression for the musician.

\subsubsection{Acknowledgements}

I would like to thank my mom..
% First and foremost, I want to wholeheartedly like to thank my supervisor Stefania Serafin for essentially my entire PhD project. I quickly found out that this project was a ``golden ticket'' to do whatever as long as it is publishable. You provided this opportunity and let me be free in my decisions throughout my project and I am extremely grateful for that.

% Both Stefan Bilbao and his seminal work \textit{Numerical Sound Systhesis} have been invaluable to the result of this project...
% Michele Ducceschi

% Other guys at Edinburgh: Brian, Charlotte, Craig

% Marius, Pelle, Nikolaj, Razvan, Ali, Lars, Nicklas, Karolina, Nathalie, Anca, Mauro

% James, Jerome, Federico Fontana, Romain, 

% Finally my family 

% Anna, who was always willing to sit through try-out lectures and conference presentations and 

% headaches from unfinished (and finished) implementations. 

\vfill
\hfill Silvin Willemsen

\hfill Aalborg University, \today
