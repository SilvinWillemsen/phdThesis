% mainfile: ../master.tex
\chapter*{Abstract\markboth{Abstract}{Abstract}}\label{ch:Abstract}
\addcontentsline{toc}{chapter}{Abstract}
Digital musical instruments exist in large quantities and numerous strategies to virtualise traditional instruments have been developed. Although one could create digital musical instruments using pre-recorded samples of their real-life counterparts, the playability and interaction of the instruments will not be captured. Instead, a simulation of the underlying physics of the instrument could be created, and is much more flexible to player interaction. This \textit{physical model} will allow a musician to be much more expressive when playing the digital instrument than if static samples were to be used. Using ad hoc hardware to control the simulation could potentially make the simulated instrument feel identical to the original.

% Physical models can be used to simulate traditional musical instruments that are too rare or valuable to be played. These cases 

% Furthermore, 
Applications of physical modelling for musical instruments include simulating instruments that are unplayable as they are too rare or vulnerable. A model of the underlying physics of the instrument could potentially resurrect the instrument making it available to the public again. 
Furthermore, as a simulation is not restricted by the laws of physics, one could extend the possibilities of the original instrument. Properties such as the material or geometry of an instrument could be dynamically changed which broadens the range of expression of the musician. One could even imagine physically impossible musical instruments which still exhibit a natural sound due to the underlying models.

In this project, finite-difference time-domain FDTD methods have been chosen, as they have an advantage in terms of generality and flexibility regarding the systems they can model. A drawback of these methods is that they are quite computationally expensive, and although many highly accurate models based on these methods have existed for years, the computing power to run them in real time has only recently become available. The main challenge %introduced by these methods 
% and physical modelling in general are: 
% \begin{enumerate}
%     \item the underlying physical model needs to be formulated, which gets increasingly complex for higher accuracy, and
%     \item  
is thus to run the simulations in real time to allow for proper player interaction.

This work presents the development and real-time implementation of various physical models of traditional musical instruments based on FDTD methods. These instruments include the trombone, the violin and more obscure instruments such as the hurdy gurdy and the tromba marina. Furthermore, a novel method is presented that paves the way for dynamic parameters in FDTD-based musical instrument simulations allowing for physically impossible instrument manipulations. Finally, this work doubles as an aid for beginners in the field of musical instrument simulations based on FDTD methods, and aims to provide a low-entry-level explanation of the literature and theory that the physical models are based on. 

\chapter*{Resum{\'e}\markboth{Resum{\'e}}{Resum{\'e}}}\label{ch:Resume}
\addcontentsline{toc}{chapter}{Resum{\'e}}
Der findes et enormt antalt digitale instrumenter og der findes adskillige strategier til virtualisering af traditionelle instrumenter. Selvom man kunne skabe digitale musikinstrumenter ved hj{\ae}lp af lydoptagelser af deres virkelige modstykke, ville instrumenternes spilbarhed og interaktion ikke blive fanget i processen. I stedet kunne man implementere en simulering af instrumentets underliggende fysik, hvilken ville v{\ae}re mere fleksibel ift. spillerinteraktion. Denne fysiske model ville g{\o}re det muligt for en musiker at v{\ae}re mere udtryksfuld n{\aa}r han eller hun spiller det digitale instrument, end hvis der bruges statiske lydoptagelser. Brug af ad hoc hardware til at styre simuleringen kunne antageligt f{\aa} det simulerede instrument til at f{\o}les identisk med originalen.
Anvendelser af fysisk modellering af musikinstrumenter inkluderer at simulere instrumenter der ikke kan spilles, da de er for sj{\ae}ldne eller s{\aa}rbare. En model af instrumentets underliggende fysik kunne potentielt genoplive instrumentet og g{\o}re det tilg{\ae}ngeligt for offentligheden igen. Ydermere kunne man forbedre det originale instrument, eftersom en simulering ikke er begr{\ae}nset af fysikkens love. Egenskaber som instruments materiale eller geometri kunne {\ae}ndres dynamisk og udvide musikerens udtryksmuligheder. Man kunne endda forestille sig fysisk umulige musikinstrumenter, der stadig udviser en naturlig lyd p{\aa} grund af de underliggende modeller.
Der findes mange fysiske modelleringsteknikker, hvor FDTD (finite-difference time-domain) metoder har en fordel med hensyn til generalitet og fleksibilitet, samt til de systemer de kan modellere. En ulempe ved disse metoder er at de er beregningstunge, og selvom der har eksisteret n{\o}jagtige modeller baseret p{\aa} disse metoder i {\aa}revis, er regnekraften til at k{\o}re dem i realtid f{\o}rst blevet tilg{\ae}ngelig for nylig. Den st{\o}rste udfordring er s{\aa}ledes at k{\o}re simuleringerne i realtid, s{\aa}dan at man kan opn{\aa} en livagtig interaktion for ud{\o}veren.			
Dette afhandling gennemg{\aa}r udviklingen og realtidsimplementeringen af forskellige fysiske modeller af traditionelle musikinstrumenter baseret p{\aa} FDTD metoder. Disse instrumenter inkluderer trombone, violin og mindre alment kendte instrumenter s{\aa}som drejelire og tromba marina. Desuden pr{\ae}senteres en ny metode, der baner vejen for dynamiske parametre i FDTD-baserede musikinstrumentsimuleringer, der muligg{\o}r instrumentmanipulationer ellers umulige i den virkelige verden. Endelig fungerer denne afhandling som et hj{\ae}lpemiddel til begyndere inden for simuleringer af musikinstrumenter baseret p{\aa} FDTD-metoder, og sigter mod at give en letford{\o}jelig forklaring af den litteratur og teori, som de fysiske modeller er baseret p{\aa}.
