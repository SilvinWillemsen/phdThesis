% mainfile: ../master.tex
\chapter*{Abstract\markboth{Abstract}{Abstract}}\label{ch:Abstract}
\addcontentsline{toc}{chapter}{Abstract}
Physical modelling %, along with sampling synthesis (also known as wavetable synthesis \cite{Smith2010a}) and spectral modelling, 
is a technique \SWcomment[/approach] to synthesise sound. It is widely accepted that physical modelling is the best way to realistically and naturally simulate real-world musical instruments \cite{Valimaki2006, Smith2010b, theBible}. As this technique simulates the instrument based on its physics rather than using pre-recorded samples, it is more flexible to player-interaction and thus more realistic when synthesising sound in performance. Although physical models could potentially sound indistinguishable from the instrument that they are simulating, it has been impossible, until recently, to make highly physically accurate physical models ‘playable’ in real-time \cite{Smith2010a}. With the computational power we currently possess, we can run the simulations in real-time and make them available for musicians in latency-less applications. These applications include digital instrument plug-ins that can be used by music producers, but also resurrect old or rare instruments that can not be played anymore due to damage, or because they are too valuable. \todo{directly copy-pasted}

\chapter*{Resumé\markboth{Resumé}{Resumé}}\label{ch:Resume}
\addcontentsline{toc}{chapter}{Resumé}
Danish Abstract
