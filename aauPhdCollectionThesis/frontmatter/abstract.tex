% mainfile: ../master.tex
\chapter*{Abstract\markboth{Abstract}{Abstract}}\label{ch:Abstract}
\addcontentsline{toc}{chapter}{Abstract}
Over the past few decades, numerous strategies to virtualise traditional instruments have been developed. Although one could create digital musical instruments using pre-recorded samples of their real-life counterparts, the playability will not be captured. Instead, a simulation of the underlying physics of the instrument could be created, and is much more flexible to player interaction. This \textit{physical model} will allow a musician to be much more expressive when playing the digital instrument than if static samples were to be used. Using ad hoc hardware to control the simulation could potentially make the simulated instrument feel identical to the original.

% Physical models can be used to simulate traditional musical instruments that are too rare or valuable to be played. These cases 

% Furthermore, 
Applications of physical modelling for musical instruments include simulating instruments that are unplayable as they are too rare or vulnerable. A model of the underlying physics of the instrument could potentially resurrect the instrument making it available to the public again. 
Furthermore, as a simulation is not restricted by the laws of physics, one could extend the possibilities of the original instrument. Properties such as the material or geometry of an instrument could be dynamically changed which broadens the range of expression of the musician. One could even imagine physically impossible musical instruments which still exhibit a natural sound due to the underlying models.

In this project, finite-difference time-domain (FDTD) methods have been chosen, as they have an advantage in terms of generality and flexibility regarding the systems they can model. A drawback of these methods is that they are quite computationally expensive, and although many highly accurate models based on these methods have existed for years, the computing power to run them in real time has only recently become available. The main challenge %introduced by these methods 
% and physical modelling in general are: 
% \begin{enumerate}
%     \item the underlying physical model needs to be formulated, which gets increasingly complex for higher accuracy, and
%     \item  
is thus to run the simulations in real time to allow for proper player interaction.

This thesis presents the development and real-time implementation of various physical models of traditional musical instruments based on FDTD methods. These instruments include the trombone and more obscure instruments such as the hurdy gurdy and the tromba marina. Furthermore, a novel method is presented that paves the way for dynamic FDTD-based musical instrument simulations allowing for physically impossible instrument manipulations. Finally, this work doubles as an aid for beginners in the field of musical instrument simulations based on FDTD methods, and aims to provide a low-entry-level explanation of the literature and theory that the physical models are based on. 

\chapter*{Resum{\'e}\markboth{Resum{\'e}}{Resum{\'e}}}\label{ch:Resume}
\addcontentsline{toc}{chapter}{Resum{\'e}}
I l{\o}bet af de sidste {\aa}rtier, er der blevet udviklet adskillige strategier til at lave virtuelle udgaver af traditionelle musikinstrumenter. Selvom digitale musikinstrumenter baseret p{\aa} traditionelle musikinstrumenter, kan skabes ved hj{\ae}lp af lydoptagelser af deres virkelige modstykke, er det ofte p{\aa} bekostning instrumenternes spilbarhed.
En anden strategi ville v{\ae}re at implementere en digital simulering af instrumentets underliggende fysik, hvilket ville give en mere fleksibel og naturlig interaktion. Denne digitale simulering, en fysisk model af instrumentet, ville g{\o}re det muligt for en musiker at v{\ae}re mere udtryksfuld n{\aa}r han eller hun spiller p{\aa} det digitale musikinstrument end med statiske lydoptagelser. Derudover, ved at bruge ad hoc hardware til at styre simuleringen, kunne man potentielt f{\aa} det digitale musikinstrument til at f{\o}les identisk med originalen. 

Fysisk modellering af musikinstrumenter kan ogs{\aa} anvendes til at simulere musikinstrumenter, der er sj{\ae}ldne eller for s{\aa}rbare til at m{\aa} spilles p{\aa}. Her ville en model af instrumentets underliggende fysik potentielt kunne genoplive instrumentet ved g{\o}re det tilg{\ae}ngeligt og spilbart igen. Ydermere, kunne man forbedre det originale instrument, eftersom en digital simulering ikke er begr{\ae}nset af fysikkens love. Egenskaber som instruments materiale eller geometri kunne dynamisk {\ae}ndres og udvide musikerens udtryksmuligheder. Man kunne endda forestille sig fysisk umulige musikinstrumenter, der stadig har en naturlig klang p{\aa} grund af de underliggende modelleringsprincipper. 

Til dette projekt er finite-difference time-domain (FDTD) metoderne blevet valgt som modelleringsteknik, siden disse metoder er generelle og fleksible og derfor har en fordel i forhold til de forskellige typer af systemer som de kan modellere.  En ulempe ved FDTD metoderne er at de er beregningstunge, og selvom der har eksisteret n{\o}jagtige modeller baseret p{\aa} disse metoder i {\aa}revis, er computer regnekraften til at k{\o}re dem i realtid f{\o}rst blevet tilg{\ae}ngelig for nyligt. Den st{\o}rste udfordring er derfor at k{\o}re simuleringerne i realtid og at opn{\aa} naturlig interaktion imellem ud{\o}veren og simuleringen. 

Denne afhandling beskriver udviklingen og implementeringen af forskellige fysiske modeller af traditionelle musikinstrumenter baseret p{\aa} FDTD metoder i realtid. Disse instrumenter inkluderer trombone og mindre kendte instrumenter som drejelire og tromba marina. Desuden pr{\ae}senteres en ny metode, der muligg{\o}r dynamiske parametre i FDTD-baserede musikinstrumentsimuleringer og tillader instrumentmanipulationer som er umulige i den virkelige verden. 
Derudover, kan denne afhandling bruges som et hj{\ae}lpemiddel til begyndere inden for simuleringer af musikinstrumenter, og sigter mod at give en begyndervenlig forklaring af den litteratur og teori, som de fysiske modeller er baseret p{\aa}. 
