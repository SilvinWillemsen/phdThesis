\chapter{An Introduction to FDTD Methods}
This chapter introduces some important concepts needed to understand the physical models presented later on in this document. 
By means of a simple mass-spring system and the 1D wave equation, the notation (and terminology) used throughout this document will be explained, together with some important analysis techniques. 
Before we dive into the mathematics, let us go over some useful terminology.

\subsubsection{Differential equations}
As mentioned in Chapter \ref{ch:physMod} differential equations are used to describe the motion of dynamic systems. A characteristic feature of these equations is that, rather than the absolute position (or displacement) of an object, the time derivative of its position -- its velocity -- or the second-order time derivative -- its acceleration -- is described. From this, the displacement of the system can be computed.

This displacement is usually described by the letter $u$ which is (nearly) always a function of time, i.e., $u=u(t)$. If the system is distributed in space, $u$ also becomes a function of space, i.e., $u = u(x,t)$, or with two spatial dimensions, $u = u(x,y,t)$, etc. Though this work only describes systems of up to two spatial dimensions, one could potentially extend to systems of infinite spatial dimensions evolving over time! 

If $u$ is only a function of time, the differential equation that describes the motion of this system is called an \textit{ordinary differential equation} (ODE). If $u$ is also a function of at least one spatial dimension, the equation of motion is a called a \textit{partial differential equation} (PDE).

The literature uses different types of notation for taking (continuous-time) partial derivatives. Applied to a state $u$ these can look like 
%
\begin{equation}\nonumber
    \begin{aligned}
        \frac{\partial^2 u}{\partial t^2} & \quad \text{(classical notation)}\\
        u_{tt}\:\,& \quad \text{(subscript notation)}\\[3pt]
        \partial^2_t u\: & \quad \text{(operator notation)}
    \end{aligned}
\end{equation}
%
all of which mean a second-order derivative with respect to time $t$, i.e., $u$'s acceleration. In this document, the operator notation will be used.

\textit{Note: difference between 1D, 2D spatial, and 1D, 2D displacement (polarisation) }

Now that an equation has been established, how 


\subsubsection{Discretisation using FDTD methods}
Finite-difference time-domain (FDTD) methods essentially subdivide a continuous equation in discrete points in time and space, a process called \textit{discretisation}.

Once an ODE or PDE is discretised using these methods it is now called a \textit{Finite-Difference Scheme} (FDS).

\subsubsection{Implementation}



In the following 
\begin{itemize}
    \item Continuous-time
    \item Discrete-time
    \item Implementation (update equation)
\end{itemize}

\section{%Intro to ODEs: 
The Mass-Spring System}
Though a complete physical modelling field on its own (see Chapter \ref{ch:physMod}), mass-spring systems are also sound-generating systems themselves.

\subsubsection{Continuous-time}
The ODE of a simple mass-spring system is defined as
\begin{equation}\label{eq:massSpringPDE}
    \frac{d^2u}{dt^2} = -\omega_0^2u
\end{equation}

\subsubsection{Discrete-time}
The , $u$ is approximated using 
\begin{equation}
    u(t) \approx u^n
\end{equation}
where $t = nk$

\subsubsection{Implementation}


\section{%Intro to PDEs: 
The 1D Wave Equation}
The simplest and arguably the most important PDE in the field is the 1D wave equation

% Note that the $\partial$ symbol is used rather than the $d$ as in \eqref{eq:massSpringPDE} as it


\subsubsection{Continuous-time}
The state of the system $u=u(x,t)$ meaning that is on top of being defined in time $t$ it is distributed over space $x$. The 1D wave equation is defined as follows
\begin{equation}
    \partial^2_t u = c^2 \partial^2_x u.
\end{equation}


\subsubsection{Discrete-time}

\subsubsection{Boundary Conditions}
When a system is distributed in space, 


\subsubsection{Output sound}
After the system is excited (see \ref{part:exciters}), one can listen to the output


\section{Energy Analysis}
Debugging physical models.

\section{Stability Analysis}
Finding stability condition
