\chapter{Conclusions and Perspectives}\label{ch:conclusion}
\todo{title is the exact same as \cite{theBible}}
Both the good and bad thing about physical modelling musical instruments is that you are never done..

There is always more work to be done

\section{Applications}
The implementations presented in this work, as well as real-time physical models of musical instruments in general, have several applications in the real world. First of all, the implementations can be used as audio plug-ins that music producers could use if they do not have


The now-digital musical instrument 

What a keyboard is to a piano, many other controllers inspired by their real-life counterpart could be made to trigger physical models of their respective instrument.
\SWcomment[Put this work into perspective of the literature (higher level)]
\section{Realism}
We are not there yet.. 

Physical modelling is not here to replace the original instruments and the musicians playing them. Instead, it can be used as a tool to understand the physics of existing instruments and possibly go beyond. Simulated instruments are not restricted by physics anymore and could provide new ways of expression for the musician. 

\subsubsection{Parameter design}
Many parameters that can always be improved

Possible perspectives could be machine learning based on audio files (literature...)

\section{Dynamic grid}
In this work, the dynamic grid in Chapter \ref{ch:dynamicGrid} the test case of the 1D wave equation has been explored. 