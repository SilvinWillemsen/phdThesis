\chapter{Modelled Excitations}\label{ch:bowModels}

\section{Hammer}
Hammer modelling

\section{The Bow}
The bow...

\subsection{Static Friction Models}
In static bow-string-interaction models, the friction force is defined as a function of the relative velocity between the bow and the string only.
The first mathematical description of friction was proposed by Coulomb in 1773 \cite{Coulomb} to which static friction, or \textit{stiction}, was added by Morin in 1833 \cite{Morin1833} and viscous friction, or velocity-dependent friction, by Reynolds in 1886 \cite{Reynolds1886}. In 1902, Stribeck found a smooth transition between the static and the coulomb part of the friction curve now referred to as the Stribeck effect \cite{Stribeck1902}. The latter is still the standard for static friction models today.

\subsection{Dynamic Friction Models}
As opposed to less complex bow models, such as the hyperbolic [source] and exponential [source] models, the elasto-plastic bow model assumes that the friction between the bow and the string is caused by a large quantity of bristles, each of which contributes to the total amount of friction.

\section{Lip-reed}
Lip-reed model


\subsection{Coupling to Tube}