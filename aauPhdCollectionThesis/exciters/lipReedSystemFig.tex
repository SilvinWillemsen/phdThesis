\begin{figure}[ht]
    \centering
    \begin{tikzpicture}
    
    \def\radius{6}; % Radius of the string (>2!)
    \pgfmathsetmacro{\reps}{3}; % How may back-and-forths in the drawing of the springs
    \def\bowSpacing{0.2};
    \def\drawingSpacing{1.5}
    \def\bowWidth{5};
    
    \def\woodWidth{1}; %>0.3
    \def\massWidth{2};
    \def\bridgeHeight{3};
    \def\bridgeWidth{4};
    \def\cornerRadius{0.15};
    \def\stringWidth{0.2};
    \pgfmathsetmacro{\tinyRadius}{\stringWidth*0.1};
    \pgfmathsetmacro{\stringWidthMinTinyRad}{((\stringWidth-(2*\tinyRadius)))*0.5};
    
    % draw airflow
    
    %draw airflow
    \def\rightAirFlow{0}; % have the right airflow bulge (1) or not (0)
    \foreach \idx in {1,...,5}
    {
        \pgfmathsetmacro{\scaleLeft}{0.5 - 0.1 * \idx};
        \ifnum\rightAirFlow=1
            \pgfmathsetmacro{\scaleRight}{0.5 - 0.1 * \idx};
        \else
            \pgfmathsetmacro{\scaleRight}{0};
        \fi
        \begin{scope}[decoration={
            markings,
            mark=between positions 0.15 and 0.85 step 0.35
         with {\arrow{>}}}
            ]
        % \node at (0, \idx) {\scale};
         \draw [gray!40, 
         xshift=-2.5cm, 
         yshift= -\idx * 0.3cm, 
         dotted, 
         line width=0.3mm, postaction={decorate}] plot [smooth, tension = 0.5] coordinates { (0,1*\scaleLeft) (1,0.75*\scaleLeft) (2,0) (4, 0) (5, 0.75*\scaleRight) (6,1*\scaleRight)} ;
         \end{scope}
    }
    % \def\scale{0.5};
    % \draw [gray, xshift=-2.5cm, yshift=-0cm] plot [smooth, tension = 0.5] coordinates { (0,1*\scale) (1,0.75*\scale) (2,0) (4, 0) (5, 0.75*\scale) (6,1*\scale)};
    % \def\scale{0.25};
    % \draw [gray, xshift=-2.5cm, yshift=-1.2cm] plot [smooth, tension = 0.5] coordinates { (0,1*\scale) (1,0.75*\scale) (2,0) (4, 0) (5, 0.75*\scale) (6,1*\scale)};
    % \def\scale{0};
    % \draw [gray, xshift=-2.5cm, yshift=-1.5cm] plot [smooth, tension = 0.5] coordinates { (0,1*\scale) (1,0.75*\scale) (2,0) (4, 0) (5, 0.75*\scale) (6,1*\scale)};
    
    \node (r0) at ( 1.0,  -0.5 ) {}; % root
    \node (s0) at ( 1.0, 0.1 ) {}; % extreme
    \node (s1) at ( 1.6, 0.1 ) {}; % extreme

    % DRAW TREE
    \fill[fill=white] (r0.center)--(s0.center)--(s1.center);
    
    % \draw plot [smooth] coordinates {(-3, -3) (-2, 2) (-1, -3};
    \node at (0,0) [rectangle,draw, fill = white, minimum height=1cm,minimum width= \massWidth cm] (Mr) {$M_\text{r}$};
    %top
    % \draw[-] (-3, 2) -- (3, 2) node[below, midway] (top) {};
    % \draw[-] (-2, 3) -- (4, 3) node[below, midway] (top) {};
    % \draw[-] (-3, 2) -- (-2, 3) node[below, midway] (top) {};
    % \draw[-] (4, 3) -- (3, 2) node[below, midway] (top) {};
    \draw[-] (-2.5, 2.5) -- (3.5, 2.5) node[below, midway] (top) {};
    %bottom
    \draw[-] (-3, -2) -- (3, -2) node[below, midway] (bottom) {};
    \draw[-] (-2, -1) -- (4, -1) node[below, midway] (bottom) {};
    \draw[-] (-3, -2) -- (-2, -1) node[below, midway] (top) {};
    \draw[-] (4, -1) -- (3, -2) node[below, midway] (top) {};

    % draw mass
    
    \draw[-] (-1, 0.5) -- (0, 1.5) node[] (left) {};

    \draw[-] (1, 0.5) -- (2, 1.5) node[] (topRight) {};

    \draw[-] (0, 1.5) -- (2, 1.5) node[] (top) {};

    \draw[-] (2, 1.5) -- (2, 0.5) node[] (right) {};

    \draw[-] (0.5 * \massWidth, -0.5) -- (2, 0.5) node[] (bottomRight) {};

    \node[rotate = 0] at (0.5, 1) (Sr) {$S_\text{r}$};

    \def\xOffset{0.3};
    % draw spring
    \filldraw[black] (-0.5 + \xOffset, 2.5) circle (1pt) node[anchor=center](topSpring){};    
    \draw[-] (-0.5 + \xOffset, 2.5) -- (-0.5 + \xOffset, 2.3);
    \draw[-] (-0.5 + \xOffset, 2.3) -- (-0.75 + \xOffset, 2.2);
    %switched these around because of the color
    \draw[-] (-0.25 + \xOffset, 2.02) -- (-0.75 + \xOffset, 1.84);
    \draw[-] (-0.75 + \xOffset, 2.2) -- (-0.25 + \xOffset, 2.02);
    \draw[-] (-0.75 + \xOffset, 1.84) -- (-0.25 + \xOffset, 1.66);
    \draw[-] (-0.25 + \xOffset, 1.66) -- (-0.75 + \xOffset, 1.52);
    \draw[-] (-0.75 + \xOffset, 1.52) -- (-0.25 + \xOffset, 1.3);
    \draw[-] (-0.25 + \xOffset, 1.3) -- (-0.5 + \xOffset, 1.2);
    \draw[-] (-0.5 + \xOffset, 1.2) -- (-0.5 + \xOffset, 1.0);
    \filldraw[black] (-0.5 + \xOffset, 1.0) circle (1pt) node[anchor=center](bottomSpring){};    
    \node at (-1 + \xOffset, 1.75) (K) {$K$};
    
    \def\dashpotHeight{-0.25}
    % draw dashpot
    \filldraw[black] (1.5 - \xOffset, 2.5) circle (1pt) node[anchor=center](topDashPot){};
    \draw[-] (1.5 - \xOffset, 2.5) -- (1.5 - \xOffset, 1.4 - \dashpotHeight);

    \draw[-] (1.3 - \xOffset, 1.4 - \dashpotHeight) -- (1.7 - \xOffset, 1.4 - \dashpotHeight);

    \draw[-] (1.25 - \xOffset, 1.7 - \dashpotHeight) -- (1.25 - \xOffset, 1.35 - \dashpotHeight);
    \draw[-] (1.75 - \xOffset, 1.35 - \dashpotHeight) -- (1.75 - \xOffset, 1.7 - \dashpotHeight);

    \draw[-] (1.25 - \xOffset, 1.35 - \dashpotHeight) -- (1.75 - \xOffset, 1.35 - \dashpotHeight);
    \draw[-] (1.5 - \xOffset, 1.35 - \dashpotHeight) -- (1.5 - \xOffset, 1.0);

    \filldraw[black] (1.5 - \xOffset, 1.0) circle (1pt) node[anchor=center](bottomDashpot){};    
    \node at (1.0 - \xOffset, 1.75) (sigma) {$R$};
    
    
    % pressure labels
    \def\pressOffset{0.3}
    \def\backgroundOpacity{0.4}
    \node[fill = white, fill opacity=\backgroundOpacity, text opacity = 1] at (-2, -0.6 - \pressOffset) (Pm) {$P_\text{m}$};
    \node[fill = white, fill opacity=\backgroundOpacity, text opacity = 1] at (0.5, -0.8 - \pressOffset) (deltaP) {$\Delta p$};
    \node[fill = white, fill opacity=\backgroundOpacity, text opacity = 1] at (3.2, -0.6 - \pressOffset) (p) {$p(0,t)$};

    
    % y and H0
    \def\axisLineWidth{0.07};
    \draw[dashed, color = gray] (3.65, 0) -- (1.5, 0);
    \node at (4, 1.5) {$y$};
    
    \draw[->] (3.75, -1.5) -- (3.75, 1.5);
    \node at (4, 0) {$0$};
    \draw (3.75 - \axisLineWidth, 0) -- (3.75 + \axisLineWidth, 0) {};

    \draw (3.75 - \axisLineWidth, -1.5) -- (3.75 + \axisLineWidth, -1.5) {};
    \node at (4.2, -1.5) {$-H_0$};
    
    % width
    \draw[black!70] (-1, 0.6) -- (1, 0.6) {};
    \draw[black!70] (-1, 0.55) -- (-1, 0.65) {};
    \draw[black!70] (1, 0.55) -- (1, 0.65) {};
    \node at (0, 0.75) (w) {$w$};
% \begin{scope}[very thick,decoration={
%     markings,
%     mark=at position 0.5 with {\arrow{>}}}
%     ] 
%     \draw[postaction={decorate}] (-4,0)--(4,0);
% \end{scope}
    
    \end{tikzpicture}
    \caption{Lip-reed system with parameters as appears in Section \ref{sec:lipreedContinuous}. (Adapted from paper \citeP[H].)}
    \label{fig:lipSystem}
\end{figure}