\chapter*{Interactions}
The models described in Part \ref{part:resonators} already sound quite convincing on their own. However, these are just individual components that can be combined to approximate a fully functional (virtual) instrument. The following chapters will describe different ways of interaction between individual systems. Chapter \ref{ch:connections} describes ways to connect different systems and Chapter \ref{ch:collisions} describes collision interactions between models. 

$\varepsilon$

Newtons third law (action reaction)

Subscripts are needed

Somewhere in this Chapter have a section about fretting and how to generate different pitches using only one string

Interpolation and spreading operators

Using $l_\ctxt = \floor[x_\ctxt / h]$ and $\alpha_\ctxt = x_\ctxt / h - l_\ctxt$ is the fractional part the location of interest.
 
\begin{equation}
    I_0(x_\ctxt) = \begin{cases}
        1, & \text{if}\ l = l_\ctxt,\\
        0, & \text{otherwise}
    \end{cases}
\end{equation}

\begin{equation}
    I_1(x_\ctxt) = \begin{cases}
        (1-\alpha_\ctxt), & \text{if}\ l = l_\ctxt\\
        \alpha_\ctxt , & \text{if}\ l = l_\ctxt + 1\\
        0 & \text{otherwise}
    \end{cases}
\end{equation}

\begin{equation}
    I_3(x_\ctxt) = \begin{cases}
        ...
    \end{cases}
\end{equation}

The following identity is very useful when solving interactions between components:
\begin{equation}\label{eq:identityIJ}
    \langle f, J_p(x_\ctxt) \rangle_\D = I_p(x_\ctxt) f
\end{equation}