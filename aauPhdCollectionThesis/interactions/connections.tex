\chapter{Connections}\label{ch:connections}
Part \ref{part:resonators} introduced various resonators in isolation. Most traditional musical instruments are made up of several of these systems which all interact with one another

Although briefly mentioned in the previous chapter in Section \ref{sec:twoSidedCollision}, 

One can use connections to simulate a point-like damping finger on a string to create different pitches.

Pointlike $\delta(x-x_\ctxt)$ or distributed $E_\ctxt$
\subsubsection{Alternative interpretation of grid points}
Section \ref{sec:gridFunctions} gives an introduction to how a continuous 1D system is subdivided into grid points in space (see Figure \ref{fig:gridExp}) in a process called discretisation. An alternative way to see grid points after discretisation is shown in Figure \ref{fig:gridExp2}. Rather than grid `points' with a spacing $h$ between them, a continuous system is divided into grid `sections' of width $h$. This interpretation allows the `weight' of a grid point to be calculated from its material properties and geometry. Notice that boundaries have a width of $h/2$ such that the total length $L = Nh$ m.

As an example, the weight of one grid point (or now rather grid section) of a string can be calculated as $\rho A h$. The weight of one grid point of a 2D plane can be calculated as $\rho H h^2$. As these grid points interact with each other, the forces resulting from this interaction will scaled by their respective weight per grid point as will be shown in Section \ref{sec:stringPlateConnection}.

This provides a better intuition for the interactions between components shown in this part. 

\begin{figure}[h]
    \centering
    % \subfloat[If $N=5$ there are 5 sections of length $h$ and 6 grid points describing the state of the system ($l=\{0, \hdots, 5\}$).\label{fig:gridExp1}]{\includegraphics[width=0.45\textwidth]{figures/fdtd/gridExplanation.pdf}}\hspace{0.06\textwidth}
    % \subfloat[If $N$ is large (as is usually the case), The 1D system is divided into $N$ sections of length $h$ and $l=\{0, \hdots, N\}$.\label{fig:gridExp2}]{\includegraphics[width=0.45\textwidth]{figures/fdtd/gridExplanation2.pdf}}
    \includegraphics[width=0.75\textwidth]{figures/fdtd/gridFigure2.pdf}
    \caption{Alternative interpretation of the discretisation of $u(x,t)$. The continuous system is divided into $N-1$ sections of width $h$ plus $2$ sections of width $h/2$ at the boundaries. Through this interpretation, the `weight' of a grid point can be calculated from its physical parameters. \label{fig:gridExp2}}
\end{figure}

\section{Rigid connection}
The simplest connection-type is the rigid connection. 

Forces should be equal and opposite. 

As explained in Chapter \ref{ch:collisions}, it is important to consider whether an object is located `above' or `below' the other. If component $a$ is located above component $b$, and their relative displacement is defined as $\eta = a-b$, then a positive $\eta$ is going to have a negative effect on $a$ and a positive effect on $b$ and vice-versa. This is important for the signs when adding the force terms to the schemes. 

\section{Spring-like connections}

\subsection{Connection with rigid barrier (scaled)}
Consider the (scaled) 1D wave equation with an additional force term $F^n$
\begin{equation}\label{eq:1DwaveConnRigid}
    \dtt \uln = \gamma^2\dxx \uln + J_l(x_\ctxt)F^n
\end{equation}
where
\begin{equation}\label{eq:1DwaveConnRigidForce}
    F^n = -\omega_0^2\mtd\eta^n - \omega_1^4(\eta^n)^2\mtd\eta^n - 2\sigma_\times \dtd \eta^n
\end{equation}
and
\begin{equation}\label{eq:etaRigid}
    \eta^n = I_l(x_\ctxt)\uln.
\end{equation}

To obtain $F^n$, an inner product of scheme \eqref{eq:1DwaveConnRigid} needs to be taken with $J_l(x_\ctxt)$ over domain $\D$ which, using identity \eqref{eq:identityIJ} yields 
\begin{equation}\label{eq:1DwaveConnRigidInnerProd}
    \dtt I_l(x_\ctxt)\uln = \gamma^2 I_l(x_\ctxt)\dxx \uln + \underbrace{I_l(x_\ctxt)J_l(x_\ctxt)}_{\lVert J_l(x_\ctxt) \rVert^2_\D}F^n.
\end{equation}
As $u$ is connected to a rigid barrier according to \eqref{eq:etaRigid}, a shortcut can be taken and Eqs. \eqref{eq:1DwaveConnRigidForce} and \eqref{eq:etaRigid} can be directly substituted into Eq. \eqref{eq:1DwaveConnRigidInnerProd} to get
\begin{equation}
    \dtt \eta^n = \gamma^2 I_l(x_\ctxt)\dxx \uln + \lVert J_l(x_\ctxt)\rVert^2_\D\left( -\omega_0^2\mtd\eta^n - \omega_1^4(\eta^n)^2\mtd\eta^n - 2\sigma_\times \dtd \eta^n\right).
\end{equation}
and solved for $\eta^{n+1}$:
\begin{equation}
    \begin{aligned}
    &\!\!\!\!\!\!\!\!\!\!\!\!\!\!\Big(1 + \lVert J_l(x_\ctxt)\rVert^2_\D k^2[\omega_0^2/2 + \omega_1^4(\eta^n)^2/2 + \sigma_\times/k] \Big)\eta^{n+1} \\
   = &\ 2 \eta^n - \Big(1 + \lVert J_l(x_\ctxt)\rVert^2_\D k^2[\omega_0^2/2 + \omega_1^4(\eta^n)^2/2 - \sigma_\times/k]\Big)
    \eta^{n-1}\\
    &+ \gamma^2k^2 I_l(x_\ctxt)\dxx\uln
    \end{aligned}
\end{equation}
This can then be used to calculate $F^n$ in \eqref{eq:1DwaveConnRigidForce} and can in turn be used to calculate $u_l^{n+1}$ in \eqref{eq:1DwaveConnRigid}.

\section{String-plate connection}\label{sec:stringPlateConnection}
In this example, let's consider a string connected to a plate using a nonlinear damped spring. This could be interpreted as a simplified form of how guitar string would be connected to the body. 

Extend section \ref{sec:interpolationSpreading} to 2D 

\subsection{Interpolation and Spreading in 2D}\label{sec:interpolationSpreading2D}
One can extend the interpolation and spreading operators presented in Section \ref{sec:interpolationSpreading} to 2D by adding an additional argument to the operators \cite{theBible}. Using $l_\itxt = \floor[x_\itxt / h]$ and $m_\itxt = \floor[y_\itxt / h]$, a $0$\thOrder interpolation operator $I_0(x_\itxt, y_\itxt) = I_{(l, m), 0}(x_\itxt, y_\itxt)$ is defined as
\begin{equation}
    I_0(x_\itxt, y_\itxt) = \begin{cases}
        1, & \text{if } l = l_\itxt \text{ and } m = m_\itxt,\\
        0, & \text{otherwise}.
    \end{cases}
\end{equation}
Notice that the same value for the grid spacing $h$ is used for both the $x$ and $y$ direction.

Using the fractional part of the flooring operations $\alpha_x = x_\itxt/h - l_\itxt$ and $\alpha_y= y_\itxt/h - m_\itxt$, a 2D linear interpolator  $I_1(x_\itxt, y_\itxt) = I_{(l, m), 1}(x_\itxt, y_\itxt)$ can then be composed of two 1D linear interpolators as
\begin{equation}
    I_1(x_\itxt, y_\itxt) = \begin{cases}
        (1 - \alpha_x)(1 - \alpha_y)& \text{if } l = l_\itxt \text{ and } m = m_\itxt, \\
        (1 - \alpha_x)\alpha_y& \text{if } l = l_\itxt \text{ and } m = m_\itxt + 1, \\
        \alpha_x(1 - \alpha_y)& \text{if } l = l_\itxt + 1 \text{ and } m = m_\itxt, \\
        \alpha_x\alpha_y& \text{if } l = l_\itxt+1 \text{ and } m = m_\itxt+1, \\
        0, & \text{otherwise}.
    \end{cases}
\end{equation}
\subsubsection{Continuous}
The systems in isolation are as in \eqref{eq:stiffStringPDE} and \eqref{eq:platePDE}, but with an added force term:
\begin{subequations}\label{eq:connStringPlatePDEs}
\begin{align}
    \ptt u &= c^2 \pxx u - \kappa_\stxt^2 \pxxxx u - 2\szX[\stxt]\pt u + 2\soX[\stxt] \pt\pxx u - \delta(x-x_\ctxt)\frac{f}{\rho_\stxt A}\\
   \ptt w &= -\kappa_\ptxt^2\Delta\Delta w - 2\szX[\ptxt]\pt w + 2\soX[\ptxt] \pt\pxx w + \delta(x-x_\ctxt, y-y_\ctxt)\frac{f}{ \rho_\ptxt H}
\end{align}
\end{subequations}
where
\begin{equation}
    f = f(t) = K_1\eta+K_3\eta^3+R \dot\eta
\end{equation}
and
\begin{equation}
    \eta = \eta(t) = u(x_\ctxt, t) - w(x_\ctxt, y_\ctxt, t)
\end{equation}

\subsubsection{Discrete}
System \eqref{eq:connStringPlatePDEs} can then be discretised as
\begin{align}\label{eq:connStringPlateFDSs}
    \!\!\!\dtt \uln &= c^2 \dxx \uln - \kappa_\stxt^2 \dxxxx \uln - 2\szX[\stxt]\dtd \uln + 2\soX[\stxt] \dtm\dxx \uln - J_\stxt(x_\ctxt)\frac{f^n}{\rho_\stxt A}\ ,\\
    \!\!\!\dtt \wlmn &= -\kappa_\ptxt^2\dDelta\dDelta \wlmn - 2\szX[\ptxt]\dtd \wlmn + 2\soX[\ptxt] \dtm\dxx \wlmn + J_\ptxt(x_\ctxt, y_\ctxt)\frac{f^n}{ \rho_\ptxt H}\ ,
\end{align}
where $J_\text{s}(x_\ctxt) = J_l(x_\ctxt)$, $J_\text{p}(x_\ctxt) = J_{l, m}(x_\ctxt)$, and
\begin{equation}
    f^n = K_1\mtt\eta^n+K_3(\eta^n)^2\mtd\eta^n+R\dtd\eta^n,
\end{equation}
and
\begin{equation}
    \eta^n = I(x_\ctxt)\uln - I(x_\ctxt, y_\ctxt)\wln.
\end{equation}

\subsubsection{Expansion}
System \eqref{eq:connStringPlateFDSs} can be expanded at the connection location $x_\ctxt$ by taking an inner product of the schemes with their respective spreading operators. 


\subsection{Solving for $f$}


\subsection{Non-dimensional}
The scaled system can be written as:
\begin{align}
    \ptt u &= \gamma^2 \pxx u - \kappa_\stxt^2 \pxxxx u - 2\szX[\stxt]\pt u + 2\soX[\stxt] \pt\pxx u - \delta(x-x_\ctxt)F\\
   \pt w &= -\kappa_\ptxt^2\Delta\Delta w - 2\szX[\ptxt]\pt w + 2\soX[\ptxt] \pt\pxx w + \delta(x-x_\ctxt, y-y_\ctxt)F
\end{align}
where
\begin{equation}
    F = F(t) = \omega_1^2\eta+\omega_3^4\eta^3+\sigma_\ctxt \dot\eta
\end{equation}
and
\begin{equation}
    \eta = \eta(t) = u(x_\ctxt, t) - w(x_\ctxt, y_\ctxt, t)
\end{equation}


\begin{equation}
    I(x_\ctxt)\delta_{tt}\uln = c^2
    \left(I(x_\ctxt)\dxx\uln\right) + I(x_\ctxt)J(x_\ctxt)F
\end{equation}