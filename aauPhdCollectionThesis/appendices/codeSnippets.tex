\chapter{Code Snippets}
\renewcommand\thesection{\thechapter.\arabic{section}}

\section{Mass-spring system (Section \ref{sec:massSpringSystem})}\label{app:massSpringCode}

\setlstMAT
\begin{lstlisting}
%% Initialise variables
fs = 44100;             % sample rate [Hz]
k = 1 / fs;             % time step [s]
lengthSound = fs;       % length of the simulation (1 second) [samples]

f0 = 440;               % fundamental frequency [Hz]
omega0 = 2 * pi * f0;   % angular (fundamental) frequency [Hz]
M = 1;                  % mass [kg]
K = omega0^2 * M;       % spring constant [N/m]

%% initial conditions (u0 = 1, d/dt u0 = 0)
u = 1;                  
uPrev = 1;

% initialise output vector
out = zeros(lengthSound, 1);

%% Simulation loop
for n = 1:lengthSound
    
    % Update equation Eq. %*\eqrefMatlab[eq:massSpringUpdate] *)
    uNext = (2 - K * k^2 / M) * u - uPrev; 
    
    out(n) = u;
    
    % Update system states
    uPrev = u;
    u = uNext;
end    
\end{lstlisting}

\section{1D wave equation (Section \ref{sec:1DWave})}
\label{app:1DWave}
\begin{lstlisting}
%% Initialise variables
fs = 44100;         % Sample rate [Hz]
k = 1 / fs;         % Time step [s]
lengthSound = fs;   % Length of the simulation (1 second) [samples]             

c = 300;            % Wave speed [m/s]
L = 1;              % Length [m]
h = c * k;          % Grid spacing [m] (from CFL condition)
N = floor(L/h);     % Number of intervals between grid points
h = L / N;          % Recalculation of grid spacing based on integer N

lambdaSq = c^2 * k^2 / h^2; % Courant number squared

% Boundary conditions ([D]irichlet or [N]eumann)
bcLeft = "D";            
bcRight = "D"; 

%% Initialise state vectors (one more grid point than the number of intervals)
uNext = zeros(N+1, 1); 
u = zeros(N+1, 1);

%% Initial conditions (raised cosine)
loc = round(0.8 * N);       % Center location
halfWidth = round(N/10);    % Half-width of raised cosine
width = 2 * halfWidth;      % Full width
rcX = 0:width;              % x-locations for raised cosine

rc = 0.5 - 0.5 * cos(2 * pi * rcX / width); % raised cosine
u(loc-halfWidth : loc+halfWidth) = rc; % initialise current state  

% Set initial velocity to zero
uPrev = u;

% Range of calculation 
range = 2:N;

% Output location
outLoc = round(0.3 * N);

%% Simulation loop
for n = 1:lengthSound
    
    % Update equation Eq. %*\eqrefMatlab[eq:1DwaveUpdate]*)
    uNext(range) = (2 - 2 * lambdaSq) * u(range) ...
        + lambdaSq * (u(range+1) + u(range-1)) - uPrev(range); 
    
    % boundary updates Eq. %*\eqrefMatlab[eq:1DWaveLeftBound]*)
    if bcLeft == "N"
        uNext(1) = (2 - 2 * lambdaSq) * u(1) - uPrev(1) ...
        + 2 * lambdaSq * u(2); 
    end

    % Eq. %*\eqrefMatlab[eq:1DWaveRightBound]*)
    if bcRight == "N"
        uNext(N+1) = (2 - 2 * lambdaSq) * u(N+1) - uPrev(N+1) ...
        + 2 * lambdaSq * u(N); 
    end
    
    out(n) = u(outLoc);
    
    % Update system states
    uPrev = u;
    u = uNext;
end
\end{lstlisting}


\section{2D wave equation (Section \ref{sec:2Dwave})}
\label{app:2DWave}
\begin{lstlisting}

%% Initialise variables
fs = 44100;         % Sample rate [Hz]
k = 1 / fs;         % Time step [s]
lengthSound = fs;   % Length of the simulation (1 second) [samples]             

rho = 7850;                     % Material density [kg/m^3]
H =  0.0005;                    % Thickness [m]
T = 1000000;                    % Tension per unit length [N/m]
c = sqrt(T / (rho * H));        % Wave speed [m/s]

Lx = 1;                         % Length in x direction [m]
Ly = 2;                         % Length in y direction [m]

h = sqrt(2) * c * k;            % Grid spacing [m]
Nx = floor(Lx/h);               % Number of intervals in x direction
Ny = floor(Ly/h);               % Number of intervals in y direction
h = min(Lx/Nx, Ly/Ny);          % Recalculation of grid spacing

lambdaSq = c^2 * k^2 / h^2;     % Courant number squared
h = max(Lx/Nx, Ly/Ny);          % Recalculation of grid spacing

%% Create scheme matrices with Dirichlet boundary conditions 
Nxu = Nx - 1;
Nyu = Ny - 1;
Dxx = toeplitz([-2, 1, zeros(1, Nxu-2)]);
Dyy = toeplitz([-2, 1, zeros(1, Nyu-2)]);

% Kronecker sum
D = kron(speye(Nxu), Dyy) + kron(Dxx, speye(Nyu));
D = D / h^2;

% Total number of grid points
Nu = Nxu * Nyu;
    
%% Initialise state vectors (one more grid point than the number of intervals)
uNext = zeros(Nu, 1); 
u = zeros(Nu, 1);

%% Initial conditions (2D raised cosine)
halfWidth = floor(min(Nx, Ny) / 5);
width = 2 * halfWidth + 1;
xLoc = floor(0.3 * Nx);
yLoc = floor(0.6 * Ny);
xRange = xLoc-halfWidth : xLoc+halfWidth;
yRange = yLoc-halfWidth : yLoc+halfWidth;

rcMat = zeros(Nyu, Nxu);
rcMat(yRange, xRange) = hann(width) * hann(width)';

% initialise current state  
u = reshape(rcMat, Nu, 1); 

% Set initial velocity to zero
uPrev = u;

% Output location
xOut = 0.45;
yOut = 0.25;
outLoc = round((xOut + yOut * Nyu) * Nxu);
out = zeros(lengthSound, 1);

%% Simulation loop
for n = 1:lengthSound
    
    %% Update equation Eq. %*\eqrefMatlab[eq:matrixUpdate2Dwave] *)
    uNext = (2 * eye(Nu) + c^2 * k^2 * D) * u - uPrev;
    
    % Update system states
    uPrev = u;
    u = uNext;
    
end
\end{lstlisting}
