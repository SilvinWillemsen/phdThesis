\chapter{Intuition for the Damping Terms in the Stiff String PDE}\label{app:intuitionSigma1}
This appendix will delve deeper into the damping terms in the PDE of the stiff string presented in Chapter \ref{ch:stiffString}, especially the frequency-dependent damping term $2\sigma_1\pt\pxx u$. Recalling the compact version of the PDE of the stiff string in Eq. \eqref{eq:stiffStringPDECompact}:
\begin{equation}
    \ptt u = c^2 \pxx u - \kappa^2 \pxxxx u - 2 \sz \pt u + 2 \so \pt \pxx u
\end{equation}
Consider first the frequency-independent damping term $-2\sz\pt u$. The more positive the velocity $\partial_tu$ is, i.e., the string is moving upwards the more this term applies a negative, or downwards force (/effect) on the string. Vice versa, a more negative velocity will make this term apply a more positive force on the string. 

As for the frequency-dependent damping term, apart from the obvious $\sigma_1$, the effect of the term increases with an increase of $\partial_t\partial_x^2u$ which literally describes the `rate of change of the curvature' of the string.

Let's first talk about positive and negative curvature, i.e., when $\partial_x^2u > 0$ or $\partial_x^2u < 0$. Counterintuitively, in the positive case, the curve points downwards. Think about the function $f(x) = x^2$. It has a positive curvature (at any point), but has a minimum. This can be proven by taking $x=0$ and setting grid spacing $h=1$.
\begin{equation}
  \begin{aligned}
  \delta_{xx}f(x) &= \frac{1}{h^2} \left(f(-1)-2f(0)+f(1)\right), \\
  &= \frac{1}{1^2} \left((-1)^2-2\cdot0^2+1^2\right),\\
  &= \left(1-0+1\right) = 2.
  \end{aligned}
\end{equation}
In other words, the second derivative of the function $f(x)=x^2$ around $x=0$ is positive.

As the term does not only include a second-order spatial derivative but also a first-order time derivative, this is referred to as a a positive or negative \textit{rate of change} of the curvature, i.e., when $\partial_t\partial_x^2u>0$ or $\partial_t\partial_x^2u<0$. A positive rate of change of curvature means that the string either has a positive curvature and is getting more positive, i.e., the string gets more curved over time, or that the string has a negative curvature and is getting less negative, i.e., the string gets less curved or 'loosens up' over time.  In the same way, a negative rate of change of curvature means that the string either has a negative curvature and is getting more negative, or that the string has a positive curvature and is getting less positive. 

Let's see some examples. Take the same function described before, but now $f$ changes over time, fx. $f(x, t)=tx^2$. When $t$ increases over time, the curvature gets bigger. Repeating the above with $x=0$ and grid spacing $h = 1$, but now with $t=2$ and step size $k=1$, but now with a backwards time derivative yields:
  \begin{alignat*}{3}
    \delta_{t-}\delta_{xx}f(x,t) &= \frac{1}{kh^2}\bigg(&&f(-1, 2) - 2f(0, 2) + f(1, 2) \\
    & &&- \Big(f(-1, 1) - 2f(0, 1) + f(1, 1)\Big)\bigg),\\
    & = \frac{1}{1\cdot 1^2}\bigg(&&2\cdot(-1)^2-2\cdot2\cdot(0)^2+2\cdot1^2\\
    & &&-\Big(1\cdot(-1)^2-2\cdot1\cdot(0)+1\cdot(1^2)\Big)\Bigg),\\
    &=2+2-&&(1+1)=2.
  \end{alignat*}
So the rate of change of the curvature is positive, i.e., the already positively curved function $x^2$ gets more curved over time.

If the curvature around a point along a string gets more positive (or less negative) over time, the force applied to that point will be positive, effectively `trying' to reduce the curvature. Vice versa, if the curvature around a point along a string gets more negative (or less positive) over time, the force applied will be negative, again `trying' to reduce the curvature. 

From an auditory point of view, higher curvature generally means higher frequency. As the frequency-dependent damping term reduces curvature along the string it effectively damps higher frequencies.
 
% The fact the frequency dependent term to be added rather than subtracted, is caused by the fact that a location along the string with a positive curvature implies that its neighbouring locations have relatively more positive displacement than itself. This translated to the force/effect this term has on the scheme means...
