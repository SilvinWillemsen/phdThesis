\chapter{Derivations}\label{app:derivations}
\def\Psilp{\Psi_{l+1}^n}
\def\Psilm{\Psi_{l-1}^n}
\def\Psinp{\Psi_l^{n+1}}
\def\Psinm{\Psi_l^{n-1}}
This appendix contains derivations of several equations used in this thesis. 

\section{Summation by parts}\label{app:summationByParts}
To see why some of the summation by parts identities in Section \ref{sec:summationByParts} hold true, it is useful to briefly go through a derivation. As an example, Eqs. \eqref{eq:summationByPartsMinus} and \eqref{eq:summationByPartsMinusBar} are derived as they have the same inner product as a starting point, but yield different results. In the following, $d=\{0, \hdots, N\}$ and $N = 2$ are used. 

Starting with Eq. \eqref{eq:summationByPartsMinus}, suppressing the $n$ superscript for brevity, and using the definition for the discrete inner product in Eq. \eqref{eq:discInnerProd}, yields
\begin{align*}
    \langle f_l, \dxm g_l \rangle_d &= \sum_{l = 0}^2 h f_l\frac{1}{h}\left(g_l - g_{l-1}\right),\\
    &= f_0g_0 - f_0g_{-1} + f_1g_1 - f_1g_0 + f_2g_2-f_2g_1,\\
    &= g_0(f_0-f_1) - f_0g_{-1} + g_1(f_1-f_2) + g_2(f_2-f_3) + f_3g_2,\\
    &= -g_0(f_1-f_0)- g_1(f_2-f_1) - g_2(f_3-f_2) + f_3g_2 - f_0g_{-1},\\
    &= -\sum_{l=0}^2 h g_l\frac{1}{h}\left(f_{l+1} - f_l\right) + f_3g_2 - f_0g_{-1},\\
    &= -\langle \dxp f_l, g_l\rangle_d + f_3g_2 - f_0g_{-1}.
\end{align*}
As $N=2$, the result is identical to Eq. \eqref{eq:summationByPartsMinus}. 

Similarly, identity \eqref{eq:summationByPartsMinusBar} can be proven to hold:
\begin{align*}
    \langle f_l, \dxm g_l \rangle_d &= \sum_{l = 0}^2 h f_l\frac{1}{h}\left(g_l - g_{l-1}\right),\\
    &= f_0g_0 - f_0g_{-1} + f_1g_1 - f_1g_0 + f_2g_2-f_2g_1,\\
    &= -f_0g_{-1} + g_0(f_0-f_1) + g_1(f_1-f_2) + f_2g_2,\\
    &= - g_0(f_1 - f_0) - g_1(f_2-f_1) + f_2g_2 - f_0g_{-1},\\
    &=\sum_{l=0}^1hg_l\frac{1}{h}(f_{l+1}-f_l) + f_2g_2 - f_0g_{-1},\\
    &= -\langle \dxp f_l, g_l\rangle_{\underline{d}} + f_2g_2 - f_0g_{-1},
\end{align*}
where the resulting inner product has a reduced domain of $\underline{d} = \{0, \hdots, N-1\}$.
Similar processes can be used to prove the other identities presented in Section \ref{sec:summationByParts}.

\section{von Neumann analysis damped stiff string}\label{app:vonNeumannString}
This section performs the von Neumann analysis presented in Section \ref{sec:stiffStringStability} in greater detail and derives the stability condition for the damped stiff string.

Starting with the characteristic equation in Eq. \eqref{eq:charDampedString}:
\begin{gather*}
    (1+\sz k)z + \left(16\mu^2\sin^4(\beta h/2)+\left(4\lambda^2+\frac{8\so k}{h^2}\right)\sin^2(\beta h/2) - 2\right)\nonumber\\
    +\left(1-\sz k-\frac{8\so k}{h^2}\sin^2(\beta h/2)\right)z^{-1}=0,
\end{gather*}
one can rewrite this to the form in Eq. \eqref{eq:polynomialForm}, and using $\S = \sin^2(\beta h /2)$ for brevity, yields
\begin{equation*}
    z^2 + \left(\frac{16\mu^2\S^2+\left(4\lambda^2+\frac{8\so k}{h^2}\right)\S - 2}{1 + \sz k}\right)z+\frac{1-\sz k-\frac{8\so k}{h^2}\S}{1 + \sz k}=0.
\end{equation*}
Stability of the system can then be proven using condition \eqref{eq:condition214}, and substituting the coefficients into this condition yields
\begin{equation}\nonumber
    \begin{aligned}
        \left|\frac{16\mu^2\S^2+\left(4\lambda^2+\frac{8\so k}{h^2}\right)\S - 2}{1+\sz k}\right|-1 &\leq \frac{1-\sz k-\frac{8\so k}{h^2}\S}{1+\sz k}\leq 1,\\
        \left|16\mu^2\S^2+\left(4\lambda^2+\frac{8\so k}{h^2}\right)\S - 2\right|-(1+\sz k) &\leq 1-\sz k-\frac{8\so k}{h^2}\S\leq 1+\sz k,\\
        \left|16\mu^2\S^2+\left(4\lambda^2+\frac{8\so k}{h^2}\right)\S - 2\right|&\leq2-\frac{8\so k}{h^2}\S\leq2+2\sz k.
    \end{aligned}
\end{equation}
The second condition is always true due to the fact that $\sigma_0,\sigma_1 \geq 0$. Continuing with the first condition: 
\begin{align*}
    -2+\frac{8\so k}{h^2}\S&\leq 16\mu^2\S^2+\left(4\lambda^2+\frac{8\so k}{h^2}\right)\S - 2\leq 2-\frac{8\so k}{h^2}\S,\\
    0&\leq 16\mu^2\S^2+4\lambda^2\S\leq 4-\frac{16\so k}{h^2}\S.
\end{align*}
As $16\mu^2\S^2+4\lambda^2\S$ is non-negative, the first condition is always satisfied. Continuing with the second condition:
\begin{align*}
    16\mu^2\S^2+\left(4\lambda^2+ \frac{16\so k}{h^2}\right)\S &\leq 4,\\
    4\mu^2\S^2+\left(\lambda^2+ \frac{4\so k}{h^2}\right)\S&\leq 1.
\end{align*}
As the left-hand side takes its maximum value for $\S= 1$, one can substitute this and continue with the substituted definitions for $\lambda$ and $\mu$ from Eq. \eqref{eq:stiffStringCourant} to yield
\begin{align*}
    \frac{4\kappa^2k^2}{h^4}+\frac{c^2k^2 + 4\so k}{h^2}&\leq 1, \\
    4\kappa^2k^2+(c^2k^2+ 4\so k)h^2&\leq h^4,\\
    h^4- (c^2k^2+ 4\so k)h^2 - 4\kappa^2k^2 &\geq 0 ,
\end{align*}
which is a quadratic equation in $h^2$. Using the quadratic formula, the grid spacing $h$ can then be shown to be bounded by
\begin{equation}
    h \geq \sqrt{\frac{c^2k^2+4\so k + \sqrt{(c^2k^2 + 4\so k)^2+16\kappa^2k^2}}{2}},
\end{equation}
which is the stability condition for the damped stiff string also shown in Eq. \eqref{eq:stiffStringStability}.

\section{von Neumann analysis implicit damped stiff string}\label{app:vonNeumannStringImplicit}
This section performs the von Neumann analysis presented in Section \ref{sec:stiffStringStabilityImplicit} in greater detail and derives the stability condition for the damped stiff string, using a centred difference operator for the frequency-dependent damping term.

Recalling the characteristic equation in Eq. \eqref{eq:charDampedImplicitStiffString}:
\begin{gather*}
    \left(1+\sz k + \frac{4\so k}{h^2}\sin^2(\beta h/2)\right)z + \left(16\mu^2\sin^4(\beta h/2)+4\lambda^2\sin^2(\beta h/2) - 2\right)\nonumber\\
    + \left(1-\sz k - \frac{4\so k}{h^2}\sin^2(\beta h/2)\right)z^{-1} = 0.
\end{gather*}
one can rewrite this to the form in Eq. \eqref{eq:polynomialForm} and, using $\S = \sin^2(\beta h / 2)$ for brevity, yields:
\begin{equation*}
z^2 + \frac{16\mu^2\S^2+4\lambda^2\S - 2}{1+\sz k + \frac{4\so k}{h^2}\S}z+\frac{1-\sz k - \frac{4\so k}{h^2}\S}{1+\sz k + \frac{4\so k}{h^2}\S} = 0.
\end{equation*}
Stability of the system can then be proven using condition \eqref{eq:condition214}, and after substitution of the coefficients yields
\begin{align*}
\left|\frac{16\mu^2\S^2+4\lambda^2\S - 2}{1+\sz k + \frac{4\so k}{h^2}\S} \right|-1 &\leq \frac{1-\sz k - \frac{4\so k}{h^2}\S}{1+\sz k + \frac{4\so k}{h^2}\S}\leq 1,\\[1em]
\left|16\mu^2\S^2+4\lambda^2\S - 2 \right| - \left(1+\sz k + \frac{4\so k}{h^2}\S\right) &\leq 1-\sz k - \frac{4\so k}{h^2}\S\\
&\qquad\leq 1+\sz k + \frac{4\so k}{h^2}\S,\\[1em]
\left|16\mu^2\S^2+4\lambda^2\S - 2 \right|&\leq 2 \leq 2+2\sz k + \frac{8\so k}{h^2}\S.
\end{align*} 
Because $\sz, \so, k, \S$ and $h$ are all non-negative, the last condition is always satisfied. Continuing with the first condition:
\begin{align*}
    -2\leq 16\mu^2\S^2+4\lambda^2\S - 2 &\leq 2,\\
    0\leq 16\mu^2\S^2+4\lambda^2\S &\leq 4.
\end{align*}
Again, the first condition is always satisfied due to the non-negativity of all coefficients. Continuing with the second condition yields
\begin{equation*}
    4\mu^2\S^2+\lambda^2\S \leq 1,
\end{equation*} 
and knowing that $\S$ is bounded by $1$ for all $\beta$, the process can be finalised:
\begin{align*}
    4\mu^2+\lambda^2 &\leq 1,\\
    \frac{4\kappa^2k^2}{h^4}+\frac{c^2k^2}{h^2} &\leq 1,\\
    h^4 - c^2k^2h^2 - 4\kappa^2k^2 &\geq 0,
\end{align*}
and yields the following stability condition:
\begin{equation*}
    h \geq \sqrt{\frac{c^2k^2 + \sqrt{c^4k^4 + 16\kappa^2k^2}}{2}}.
\end{equation*}
\section{Webster's update equation}\label{app:webstersUpdateEq}
This section derives the update equation for Webster's equation in Eq. \eqref{eq:webstersUpdateEq}:
\begin{align*}
    &\begin{aligned}
    \frac{\Sbar}{k^2}(\Psi_l^{n+1} - 2\Psiln+\Psi_l^{n-1}) &= c^2\bigg((\dxm \Sp)(\mxm \dxp \Psiln)\\
    & + (\mxm \Sp)(\dxm \dxp \Psiln)\bigg),\\
    \Psinp - 2\Psiln+\Psinm &= \frac{c^2k^2}{\Sbar}\bigg(\frac{1}{h}(\Sp - \Sm)\frac{1}{2h}\overbrace{(\Psilp -\Psilm)}^{\mxp\dxm\Psiln = \delta_{x\cdot}\Psiln}\nonumber\\
    &+\frac{1}{2}(\Sp + \Sm)\frac{1}{h^2}(\Psilp-2\Psiln+\Psilm)\bigg),
    \end{aligned}\\[-0.5em]
    &\begin{aligned}
        \Psinp &= 2\Psiln -\Psinm +\overbrace{\frac{\lambda^2}{2\Sbar}}^{\lambda = \frac{ck}{h}}\Big(\Sp\Psilp - \Sp\Psilm\\
        &\qquad\qquad - \Sm\Psilp + \Sm\Psilm+ \Sp\Psilp + \Sp\Psilm \\
        &\qquad\qquad + \Sm\Psilp + \Sm\Psilm - 2 (\Sp + \Sm)\Psiln\Big)\nonumber,\\
        \Psinp &= 2\Psiln-\Psinm+ \frac{\lambda^2}{2\Sbar}\Big(2\Sp\Psilp + 2\Sm\Psilm - 4\Sbar\Psiln\Big)\nonumber,\\
        \Psinp &= 2\Psiln-\Psinm+ \frac{\lambda^2\Sp}{\Sbar}\Psilp + \frac{\lambda^2\Sm}{\Sbar}\Psilm - 2\lambda^2\Psiln\nonumber,\\
        \Psinp &= 2(1-\lambda^2)\Psiln-\Psinm+ \frac{\lambda^2\Sp}{\Sbar}\Psilp + \frac{\lambda^2\Sm}{\Sbar}\Psilm.
    \end{aligned}
\end{align*}

\section{Boundary terms Webster's equation}\label{app:boundaryWebster}
This section derives the process of obtaining the values for $\el$ and $\er$ such that the boundary terms in Webster's equation are strictly dissipative. The result is presented in Eq. \eqref{eq:centStrictDissip} and will be derived here.

Starting with the second term in the energy balance in Eq. \eqref{eq:powerBalanceWebster}:
\begin{equation}
    - c^2\langle\dtd\Psiln, \dxm\big(\Sp(\dxp\Psiln)\big)\rangle_d^{\el,\er},
\end{equation}
and using identity \eqref{eq:weightedIdentityMinus}:
\begin{equation*}
    \begin{aligned}
    \langle f_l^n, \dxm g_l^n \rangle_d^{\el,\er}  = -&\langle \dxp f_l^n, g_l^n \rangle_{\underline{d}}+ f_N^ng_{N-1}^n - f_0^ng_0^n \\
    &+ \frac{\epsilon_\text{r}}{2}f_N^n(g_N^n-g_{N-1}^n)+ \frac{\epsilon_\text{l}}{2}f_0^n(g_0^n - g_{-1}^n),
    \end{aligned}
\end{equation*}
this can be rewritten to (with $f=\dtd\Psi$ and $g = S_{l+1/2}(\dxp\Psi)$)
\begin{equation*}
        -c^2\langle \dtd\Psiln,\dxm \big(\Sp(\dxp\Psiln)\big) \rangle_{d}^{\epsilon_\text{l},\epsilon_\text{r}} = c^2\langle \dtd\dxp \Psiln, \big(\Sp(\dxp\Psiln)\big)\rangle_{\underline{d}} - \b,
\end{equation*}
where
\begin{equation}
    \b = \b_\text{r} - \b_\text{l},
\end{equation}
with
\begin{equation*}
    \mathfrak{b}_\text{r} =  \begin{aligned}[t]
        &c^2 (\dtd\Psi_N^n)\Big(S_{N-1/2}\overbrace{(\dxp\Psi_{N-1}^n)}^{(\dxm\Psi_{N}^n)}\Big)\\
        &+ \frac{\epsilon_\text{r}}{2}(\dtd\Psi_{N}^n)\Big(S_{N+1/2}(\dxp\Psi_N^n)- S_{N-1/2}\underbrace{(\dxp \Psi_{N-1}^n)}_{(\dxm\Psi_{N}^n)}\Big),
    \end{aligned}
\end{equation*}
and 
\begin{equation*}
    \mathfrak{b}_\text{l} = \begin{aligned}[t]
        &c^2(\dtd\Psi_0^n)\Big(S_{1/2}(\dxp\Psi_0^n)\Big)\\
        & -\frac{\epsilon_\text{l}}{2}(\dtd\Psi_0^n)\Big(S_{1/2}(\dxp\Psi_0^n)-S_{-1/2}\underbrace{(\dxp\Psi_{-1}^n)}_{(\dxm\Psi_0^n)}\Big).
    \end{aligned}
\end{equation*}
This can be rewritten to
\begin{align}
    \b_\text{r} &= c^2(\dtd\Psi_N^n)\left(\frac{\epsilon_\text{r}}{2}S_{N+1/2}(\dxp \Psi_N^n) + \left(1-\frac{\epsilon_\text{r}}{2}\right)S_{N-1/2}(\dxm\Psi_N^n)\right),\\
    \b_\text{l} &= c^2(\dtd\Psi_0^n)\left(\frac{\epsilon_\text{l}}{2}S_{-1/2}(\dxm\Psi_0^n)+\left(1-\frac{\epsilon_\text{l}}{2}\right)S_{1/2}(\dxp \Psi_0^n))\right).
\end{align}
% \\
% &= \\
% &\qquad - c^2(\dtd\Psi_0^n)\left(\frac{\epsilon_\text{l}}{2}S_{-1/2}(\dxm\Psi_0^n)+\left(1-\frac{\epsilon_\text{l}}{2}\right)S_{1/2}(\dxp \Psi_0^n))\right)

Then, for the centred radiating boundary condition in Eq. \eqref{eq:centRadBound} to be strictly dissipative, i.e., $\dxd \Psiln = 0 \ \Rightarrow \ \b_\text{r} = 0$, the special choice for $\epsilon_\text{r} = S_{N-1/2} / \mu_{xx}S_N$ needs to be made:
\begin{equation}
    \begin{aligned}
        \mathfrak{b}_\text{r} &= c^2 (\dtd\Psi_N^n)\!\left(\frac{S_{N-1/2}}{2\mu_{xx}S_N}S_{N+1/2}(\dxp\Psi_N^n)+\left(1\!-\!\frac{S_{N-1/2}}{2\mu_{xx}S_N}\right)\!S_{N-1/2}(\dxm\Psi_N^n)\!\right),\\
        &= c^2 (\dtd\Psi_N^n)S_{N-1/2}\left(\frac{S_{N+1/2}}{2\mu_{xx}S_N}(\dxp\Psi_N^n)+\left(1-\frac{S_{N-1/2}}{2\mu_{xx}S_N}\right)(\dxm\Psi_N^n)\right),\\
        &=c^2 (\dtd\Psi_N^n)S_{N-1/2}\left(1-\frac{S_{N-1/2}}{2\mu_{xx}S_N}\right)\left(\frac{\frac{S_{N+1/2}(\dxp\Psi_N^n)}{2\mu_{xx}S_N}}{\left(1-\frac{S_{N-1/2}}{2\mu_{xx}S_N}\right)}+\dxm\Psi_N^n\right),\\
        &=c^2 (\dtd\Psi_N^n)S_{N-1/2}\left(1-\frac{\epsilon_\text{r}}{2}\right)\left(\frac{\frac{S_{N+1/2}(\dxp\Psi_N^n)}{2\mu_{xx}S_N}}{\left(\frac{2\mu_{xx}S_N - S_{N-1/2}}{2\mu_{xx}S_N}\right)}+\dxm\Psi_N^n\right),\\
        &=c^2 (\dtd\Psi_N^n)S_{N-1/2}\left(1-\frac{\epsilon_\text{r}}{2}\right)\left(\frac{S_{N+1/2}(\dxp\Psi_N^n)}{2\mu_{xx}S_N - S_{N-1/2}}+\dxm\Psi_N^n\right),\\
        &=c^2 (\dtd\Psi_N^n)S_{N-1/2}\left(1-\frac{\epsilon_\text{r}}{2}\right)\left(\frac{S_{N+1/2}(\dxp\Psi_N^n)}{S_{N+1/2} + S_{N-1/2} - S_{N-1/2}}+\dxm\Psi_N^n\right),\\
        &=c^2 (\dtd\Psi_N^n)S_{N-1/2}\left(1-\frac{\epsilon_\text{r}}{2}\right)\left(\dxp\Psi_N^n+\dxm\Psi_N^n\right),\\
        &=c^2 (\dtd\Psi_N^n)S_{N-1/2}\left(1-\frac{\epsilon_\text{r}}{2}\right)\left(\frac{1}{h}\left(\Psi_{N+1}^n - \Psi_N^n + \Psi_N^n - \Psi_{N-1}^n\right)\right),\\
&= c^2 (\dtd\Psi_N^n)S_{N-1/2}\left(1-\frac{\epsilon_\text{r}}{2}\right)2(\delta_{x\cdot}\Psi_N^n),\\
&= c^2 (\dtd\Psi_N^n)S_{N-1/2}\left(2-\epsilon_\text{r}\right)(\delta_{x\cdot}\Psi_N^n).
    \end{aligned}
\end{equation}
The same can be done for $\mathfrak{b}_\text{l}$ with $\epsilon_\text{l} = S_{1/2}/\mu_{xx}S_0$ to get
\begin{equation}
    \mathfrak{b}_\text{l} = c^2(\dtd\Psi_0^n)S_{1/2}\left(2-\epsilon_\text{l}\right)(\delta_{x\cdot}\Psi_0^n).
\end{equation}


\section{Levine and Schwinger radiation model update equation}\label{app:levineSchwingDeriv}
\def\r{\text{r}}
\def\one{{(1)}}
This section provides a derivation for the update equation in Eq. \eqref{eq:pRadUpdate} and follows \cite[Sec. 4.1.3, pp. 109--111]{Harrison2018}. Recalling system \eqref{eq:barVPSystem}
\begin{subequations}\label{eq:barVPSystemApp}
    \begin{align}
        \bar v &= \mu_{t+}v_\one + \frac{1}{R_2}\mu_{t+}p_\one + C_\r \delta_{t+}p_\one,\label{eq:barVApp}\\
        \bar p &= L_\r \delta_{t+}v_\one,\label{eq:barP1App}\\
        \bar p &= \left(1+\frac{R_1}{R_2}\right)\mu_{t+}p_\one+ R_1 C_\r\delta_{t+}p_\one\label{eq:barP2App},
    \end{align}
\end{subequations}
where $\bar p^{n+1/2}$ and $\bar v^{n+1/2}$ are related to the tube by (see Eq. \eqref{eq:barVars})
\begin{align}
    \bar p &= \mu_{t+}p^n_N,\label{eq:barVarP}\\
    \bar S_N \bar v &= \mu_{x-}\left(S_{N+1/2}v_{N+1/2}^{n+1/2}\right),\label{eq:barVarV}
\end{align}
one can start the derivation. 

The radiation can be applied to the right boundary of the tube by evaluating the update equation of the pressure in Eq. \eqref{eq:pressureUpdate} at $l = N$
\begin{equation}
    p_N^{n+1} = p_N^n - \frac{\rho_0 c \lambda}{\bar{S}_N}\left(S_{N+1/2}v_{N+1/2}^{n+1/2}-S_{N-1/2}v_{N-1/2}^{n+1/2}\right),
\end{equation}
rewriting this to 
\begin{equation*}
    p_N^{n+1} = p_N^n - \frac{\rho_0 c \lambda}{\bar{S}_N}\left(2\mu_{x-}\left(S_{N+1/2}v_{N+1/2}^{n+1/2}\right)-2S_{N-1/2}v_{N-1/2}^{n+1/2}\right),
\end{equation*}
and substituting Eq. \eqref{eq:barVarV} to get
\begin{equation}
    p_N^{n+1} = p_N^n - \frac{2\rho_0 c \lambda}{\bar{S}_N}\left(\bar S_N \bar v-S_{N-1/2}v_{N-1/2}^{n+1/2}\right)\label{eq:preSolutP}.
\end{equation}
A definition for $\bar v$ can then be found by first expanding Eq. \eqref{eq:barVApp} to 
\begin{equation}\label{eq:vBarExpanded}
    \bar v = \frac{1}{2}\left(v_\one^{n+1} + v_\one^n\right) + \left(\frac{1}{2R_2} + \frac{C_\r}{k}\right) p_\one^{n+1} +\left(\frac{1}{2R_2} - \frac{C_\r}{k}\right)p_\one^n,
\end{equation}
after which it should be made solely dependent on the known values $v_\one^n$, $p_\one^n$ and $p_N^n$ and the unknown $p_N^{n+1}$ (as this can be obtained using Eq. \eqref{eq:preSolutP}).

Equation \eqref{eq:barP1App} can be expanded to 
\begin{equation}\label{eq:voneNext}
    v_\one^{n+1} = \frac{k}{L_\r}\bar p + v_\one^n ,
\end{equation}
and Eq. \eqref{eq:barP2App} to 
\begin{align*}
    &\bar p =\left(1+\frac{R_1}{R_2}\right)\mu_{t+}p_\one+ R_1 C_\r\delta_{t+}p_\one,\\
    &\bar p =\frac{1}{2}\left(1+\frac{R_1}{R_2}\right)\left(p_\one^{n+1} + p_\one^n\right) + \frac{R_1C_\r}{k}\left(p_\one^{n+1} - p_\one^n\right),\\
    \left(\frac{1}{2}+\frac{R_1}{2R_2} + \frac{R_1C_\r}{k}\right)&p_\one^{n+1} = \bar p + \left(\frac{R_1C_\r}{k} - \frac{1}{2} - \frac{R_1}{2R_2}\right)p_\one^n,
\end{align*}
and finally solved for $p_\one^{n+1}$ as
\begin{equation}\label{eq:poneNext}
    p_\one^{n+1} = \underbrace{\left(\frac{2R_2k}{2R_1R_2C_\r + k(R_1 + R_2)}\right)}_{\zeta_1}\bar p + \underbrace{\left(\frac{2R_1R_2C_\r - k(R_1 + R_2)}{2R_1R_2C_\r + k(R_1 + R_2)}\right)}_{\zeta_2} p_\one^n .
\end{equation}
Equations \eqref{eq:voneNext} and \eqref{eq:poneNext} can then be substituted into Eq. \eqref{eq:vBarExpanded} and, using the definition of $\bar p$ from Eq. \eqref{eq:barVarP}, yields
\begin{align}
    \bar v &= \frac{1}{2}\left(\frac{k}{L_\r}(\mu_{t+}p_N^n) + 2v_\one^n\right)+\left(\frac{1}{2R_2} + \frac{C_\r}{k}\right)\zeta_1\mu_{t+}p_N^n\nonumber \\
    & \qquad\qquad\qquad\qquad+ \left(\frac{1}{2R_2} + \frac{C_\r}{k}\right)\zeta_2p_\one^n + \left(\frac{1}{2R_2} - \frac{C_\r}{k}\right)p_\one^n\nonumber,\\
    \bar v &= \underbrace{\left(\frac{k}{2L_\r} + \frac{\zeta_1}{2R_2}+\frac{C_\r\zeta_1}{k}\right)}_{\zeta_3}\mu_{t+}p_N^n + v_\one^n + \underbrace{\left(\frac{\zeta_2+1}{2R_2} + \frac{C_\r\zeta_2 - C_\r}{k}\right)}_{\zeta_4}p_\one^n.
\end{align}
Finally, substituting this definition for $\bar v$ into Eq. \eqref{eq:preSolutP}, yields
\begin{align*}
    p_N^{n+1} &= p_N^n\! - \!\frac{2\rho_0c\lambda}{\bar S_N}\left(\bar S_N\!
    \left[\zeta_3\left(\frac{p_N^{n+1} + p_N^n}{2}\right) + v_\one^n + \zeta_4p_\one^n\right] \!-\! S_{N-1/2}v_{N-1/2}^{n+1/2}\right),\\
    p_N^{n+1} &= p_N^n - \rho_0c\lambda\left(\zeta_3(p_N^{n+1} + p_N^n) + 2(v_\one^n + \zeta_4p_\one^n)-\frac{2S_{N-1/2}v_{N-1/2}^{n+1/2}}{\bar S_N}\right),
\end{align*}
and yields a definition for $p_N^{n+1}$ based on known values of the system
\begin{equation}\label{eq:pradApp}
    p_N^{n+1} = \frac{1 - \rho_0c\lambda\zeta_3}{1+\rho_0c\lambda\zeta_3}p_N^n - \frac{2\rho_0c\lambda}{1+\rho_0c\lambda\zeta_3} \left( v_\one^n+\zeta_4p_\one^n - \frac{S_{N-1/2}v_{N-1/2}^{n+1/2}}{\bar S_N}\right),
\end{equation}
which is Eq. \eqref{eq:pRadUpdate}. After $p_N^{n+1}$ is calculated, $v_\one^{n+1}$ and $p_\one^{n+1}$ can be updated according to Eqs. \eqref{eq:voneNext} and \eqref{eq:poneNext}, respectively.

\section{Derivatives for Newton-Raphson for the elasto-plastic friction model}\label{app:elastoDeriv}
This section provides the derivatives for the Newton-Raphson iteration for the elasto-plastic friction model in Eq. \eqref{eq:NRit}.

Recalling the functions needed to compute the Newton-Raphson iteration for the elasto-plastic bow model in Section \ref{sec:elastoPlastic}, being Eq. \eqref{eq:elastog1}:
\begin{equation*}
    g_1(v^n,z^n) = \left(\frac{2}{k} + 2\sz\right)v^n + \lVert J_l(x_\Btxt^n)\rVert_d^2\frac{f(v^n,z^n)}{\rho A}+ b^n= 0,
\end{equation*}
and Eq. \eqref{eq:elastog2}
\begin{equation*}
    g_2(v^n, z^n) = r^n - a^n = 0,
\end{equation*}
the derivatives needed to solve the Newton-Raphson iteration in Eq. \eqref{eq:NRit} 
\begin{equation*}
    \begin{bmatrix}
        v^n\\
        z^n
        \end{bmatrix}_{i+1}
        =
        \begin{bmatrix}
        v^n\\
        z^n
        \end{bmatrix}_i
        -
        \begin{bmatrix}
        \frac{\partial g_1}{\partial v} & \frac{\partial g_1}{\partial z}\\
        \frac{\partial g_2}{\partial v} & \frac{\partial g_2}{\partial z}\\
        \end{bmatrix}^{-1}
        \begin{bmatrix}
        g_1\\
        g_2
        \end{bmatrix}\,
        .
\end{equation*}
can be shown to be
\begin{align*}
    \frac{\partial g_1}{\partial v} &= \frac{2}{k} + 2\sigma_0 + \frac{s_1\lVert J_l(x_\Btxt^n)\rVert_d^2}{\rho A}\frac{\partial r}{\partial v}+\frac{s_2\lVert J_l(x_\Btxt^n)\rVert_d^2}{\rho A},\\
    \frac{\partial g_1}{\partial z} &= \frac{s_0\lVert J_l(x_\Btxt^n)\rVert_d^2}{\rho A} + \frac{s_1\lVert J_l(x_\Btxt^n)\rVert_d^2}{\rho A}\frac{\partial r}{\partial z},\\
    \frac{\partial g_2}{\partial v} &= \frac{\partial r}{\partial v}
    \\\frac{\partial g_2}{\partial z}&= \frac{\partial r}{\partial z} -\frac{2}{k}.
\end{align*}
Recalling from Eq. \eqref{eq:r} that
\begin{equation*}
    r^n = r(v^n,z^n) = v^n\bigg[1-\alpha(v^n,z^n)\frac{z^n}{z_\text{ss}(v^n)}\bigg],
\end{equation*}
its derivatives can be computed as
\begin{align*}
    \frac{\partial r}{\partial v} &= 1-z^n\Bigg(\frac{(\alpha^n+\frac{\partial \alpha^n}{\partial v}v^n)z_\text{ss}^n - \frac{\partial z_\text{ss}^n}{\partial v}\alpha^n v^n}{(z_\text{ss}^n)^2}\Bigg),\\
    \frac{\partial r}{\partial z} &= -\frac{v^n}{z_\text{ss}^n}\bigg(\frac{\partial \alpha^n}{\partial z}z^n + \alpha^n\bigg),
    \end{align*}
with
\begin{align*}
    \frac{\partial\alpha^n}{\partial v} &=\sgn(z_\text{ss}^n)\frac{\partial z_\text{ss}^n}{\partial v}\frac{z_\text{ba} - |z^n|}{(|z_\text{ss}^n| - z_\text{ba})^2}\frac{\pi}{2}\cos\big(\sgn(z^n)\Phi\big),\\
    \frac{\partial\alpha^n}{\partial z}&=\frac{\sgn(z^n)\pi\cos\big(\sgn(z^n)\Phi\big)}{2(|z_\text{ss}^n|-z_\text{ba})},\\
    \frac{\partial z_\text{ss}^n}{\partial v} &= -\frac{2|v^n|}{v_\text{S}^2 s_0}(f_\text{S}-f_\text{C})e^{-(v^n/v_\text{S})^2}.
\end{align*}
